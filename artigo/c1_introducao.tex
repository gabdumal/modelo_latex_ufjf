\section{Introdução}%
\label{sec:introducao}

Este modelo se baseia no Modelo Canônico para trabalhos Acadêmicos do projeto \abnTeX~\cite{abntex2:2024}.

Também se utilizam recursos desenvolvidos pelo projeto de \citeonline{souza:2024}, que adapta as regras de formatação da \gls{ufjf} para um modelo em \LaTeX.

Por fim, as definições foram ajustadas conforme o Manual de normalização e modelos de trabalhos acadêmicos da \gls{ufjf}~\cite{cdd:2023}.

\section{Normas}%
\label{sec:normas}

As seguintes normas foram seguidas para a elaboração deste trabalho:

\begin{description}
    \item[NBR 6022:2018] Informação e documentação --- Artigo em publicação periódica técnica e/ou científica --- Apresentação~\cite{nbr6022:2018}
    \item[NBR 6027:2012] Informação e documentação --- Sumário --- Apresentação~\cite{nbr6027:2012}
    \item[NBR 6028:2021] Informação e documentação --- Resumo, resenha e recensão --- Apresentação~\cite{nbr6028:2021}
    \item[NBR 14724:2011] Informação e documentação --- Trabalhos acadêmicos --- Apresentação~\cite{nbr14724:2011}
\end{description}

A norma NBR 6028:2021~\cite{nbr6028:2021} define que as palavras-chave devem ser separadas pelo caractere ponto e vírgula e finalizadas por ponto.
Além disso, elas devem ser apresentadas em letra minúscula, salvo siglas e nomes próprios.
