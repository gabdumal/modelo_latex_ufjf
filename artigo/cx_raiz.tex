\documentclass[
    % -- opções da classe memoir --
    article,			% indica que é um artigo acadêmico
    11pt,				% tamanho da fonte
    oneside,			% para impressão apenas no recto. Oposto a twoside
    a4paper,			% tamanho do papel. 
    % -- opções da classe abntex2 --
    %chapter=TITLE,		% títulos de capítulos convertidos em letras maiúsculas
    %section=TITLE,		% títulos de seções convertidos em letras maiúsculas
    %subsection=TITLE,	% títulos de subseções convertidos em letras maiúsculas
    %subsubsection=TITLE % títulos de subsubseções convertidos em letras maiúsculas
    % -- opções do pacote babel --
    english,			% idioma adicional para hifenização
    brazil,				% o último idioma é o principal do documento
    sumario=tradicional
]{abntex2}

% Pacotes
% -----
% Pacotes fundamentais 
% -----

\usepackage{lmodern}        % Usa a fonte Latin Modern.
\usepackage[T1]{fontenc}    % Seleção de códigos de fonte.
\usepackage[utf8]{inputenc} % Codificação do documento (conversão automática dos acentos).
\usepackage{indentfirst}    % Identa o primeiro parágrafo de cada seção.
\usepackage{microtype}      % Para melhorias de justificação.

\usepackage{color}      % Controle das cores.
\usepackage{graphicx}   % Inclusão de gráficos.

\usepackage[brazilian,hyperpageref]{backref}    % Páginas com as citações na bibliografia.

\usepackage[subentrycounter,seeautonumberlist,nonumberlist=true,acronym,nohypertypes={index}]{glossaries}   % Glossário. Permite definir termos, siglas e abreviações. Deve ser carregado depois de backref.

\usepackage[alf]{abntex2cite}       % Citações padrão ABNT. Deve ser carregado depois de glossaries.

\usepackage[portuguese]{todonotes}  % Adiciona notas e afazeres no documento.

\usepackage{multirow}   % Permite mesclar células em tabelas.
\usepackage{longtable}  % Permite criar tabelas com mais de uma página.

\usepackage{lipsum}     % Gera texto de preenchimento.

% -----

% -----
% Configurações dos pacotes
% -----

% --- Glossaries ---
\newglossary[ilg]{index}{ind}{idx}{\indexname}
\newcommand*{\newterm}[2]{
    \newglossaryentry{#1}
    {type=index,name={#2},description={\nopostdesc}}
}
\makeglossaries{} % Habilite este comando para permitir a impressão dos glossários.
\loadglsentries{c0_glossario.tex}
\renewcommand*{\glsclearpage}{} % Evita quebra de página entre os glossários.

% --- Backref ---
% Usado sem a opção hyperpageref de backref.
\renewcommand{\backrefpagesname}{Citado na(s) página(s):~}
% Texto padrão antes do número das páginas.
\renewcommand{\backref}{}
% Define os textos da citação.
\renewcommand*{\backrefalt}[4]{%
    \ifcase #1%
        Nenhuma citação no texto.%
    \or%
        Citado na página #2.%
    \else%
        Citado #1 vezes nas páginas #2.%
    \fi}%

% --- Todonotes ---
% Define a largura da caixa de notas.
\setlength{\marginparwidth}{2cm}
% Define a cor e o estilo da caixa de notas.
\presetkeys{todonotes}{inline,backgroundcolor=yellow}{}
% Desabilita as notas.
% \presetkeys{todonotes}{disable}{}

% -----


% Definições da Capa
% -----
% Informações de dados para CAPA e FOLHA DE ROSTO
% -----
\titulo{Título}
\tituloestrangeiro{Title}

\autor{%
    Autor Um \and
    Autor Dois
}

\local{Juiz de Fora}
\data{\glsentrydesc{dcc}, \glsentrydesc{ufjf}\newline[INSIRA O ANO AQUI]}
% -----

% Configurações do documento
% -----
% Configurações de aparência do PDF final
% -----
\definecolor{blue}{RGB}{41,5,195}
% Informações do PDF
\makeatletter
\hypersetup{%
    %pagebackref=true,
    pdftitle={\ValorDoTitulo},
    pdfauthor={\@author},
    pdfsubject={Modelo de artigo científico com abnTeX2},
    pdfcreator={LaTeX with abnTeX2},
    pdfkeywords={abnt}{latex}{abntex}{abntex2}{artigo científico},
    colorlinks=true,       		% false: boxed links; true: colored links
    linkcolor=blue,          	% color of internal links
    citecolor=blue,        		% color of links to bibliography
    filecolor=magenta,      		% color of file links
    urlcolor=blue,
    bookmarksdepth=4
}
\makeatother
% -----

% -----
% Demais configurações
% -----

% Compila o índice
\makeindex

% Altera as margens padrões
\setlrmarginsandblock{3cm}{3cm}{*}
\setulmarginsandblock{3cm}{3cm}{*}
\checkandfixthelayout{}

% Espaçamentos entre linhas e parágrafos
\setlength{\parindent}{1.3cm} % Tamanho do parágrafo
\setlength{\parskip}{0.2cm}  % Controle do espaçamento entre um parágrafo e outro
\SingleSpacing{} % Espaçamento simples

% -----

% Início do documento
\begin{document}

% -----
% Configurações do texto
% -----
% Seleciona o idioma do documento
\selectlanguage{brazil}

% Retira espaço extra obsoleto entre as frases
\frenchspacing{}
% -----

% =====
% ELEMENTOS PRÉ-TEXTUAIS
% =====
\pretextual{}

% Página de titulo principal (obrigatório)
\maketitle{}

% Resumos
% --- Português ---
\begin{resumoumacoluna}

    % Escreva aqui o resumo em português
    \lipsum[1-3]

    \vspace{\onelineskip}

    \noindent
    % Escreva aqui as palavras-chave em português
    \textbf{Palavras-chave}: \ValorDasPalavrasChave{}.
\end{resumoumacoluna}

% --- Língua estrangeira ---
\renewcommand{\resumoname}{Abstract}
\begin{resumoumacoluna}
    \begin{otherlanguage*}{english}

        % Write here the abstract in English
        \lipsum[4-6]

        \vspace{\onelineskip}

        \noindent
        % Write here the keywords in English
        \textbf{Keywords}: word one, word two, word three, word four, word five.
    \end{otherlanguage*}
\end{resumoumacoluna}


\begin{center}\smaller{}
    \textbf{Data de submissão e aprovação}: [INSIRA A DATA AQUI].
\end{center}
% =====

% =====
% ELEMENTOS TEXTUAIS
% =====
\textual{}

\chapter{Introdução}%
\label{cap:introducao}

Este modelo se baseia no Modelo Canônico para trabalhos Acadêmicos do projeto \abnTeX~\cite{abntex2:2024}.

Também se utilizam recursos desenvolvidos pelo projeto de \citeonline{souza:2024}, que adapta as regras de formatação da \gls{ufjf} para um modelo em \LaTeX.

Por fim, as definições foram ajustadas conforme o Manual de normalização e modelos de trabalhos acadêmicos da \gls{ufjf}~\cite{cdd:2023}.


\section{Fundamentação Teórica}%
\label{sec:fundamentacao}

Apresentamos exemplos de uso de símbolos matemáticos e equações.

\subsection{Dados}

\subsubsection{Índices}

O problema permite delimitar os seguintes índices:

\begin{symbols}
    \item[\( \gls{indice:receitas} \)]
    \glsentrydesc{indice:receitas},
    tal que \(  \gls{indice:receitas} \in [1, N] \);

    \item[\( \gls{indice:pacotes} \)]
    \glsentrydesc{indice:pacotes},
    tal que \(  \gls{indice:pacotes} \in [1, N] \);

    \item[\( \gls{indice:pedidos} \)]
    \glsentrydesc{indice:pedidos},
    tal que \(  \gls{indice:pedidos} \in [1, N] \);
\end{symbols}

\subsubsection{Constantes}

\subsubsubsection{Fixas}

As seguinte constante é fixa e não pode ser alterada durante a execução do modelo.

\begin{symbols}
    \item[\( \gls{constante:fixa:tempo-turno} \) ]
    \glsentrydesc{constante:fixa:tempo-turno},
    tal que \( \gls{constante:fixa:tempo-turno} \in \mathbb{N}^{+} \).
\end{symbols}

O valor da constante citada e sua unidade são exibidos no \autoref{qua:constantes-fixas}.

\begin{quadro}
    \caption{%
        \label{qua:constantes-fixas}%
        Constantes do problema.
    }
    \begin{tabular}{|c|c|c|}
        \hline
        Constante                              &
        Valor                                  &
        Unidade
        \\
        \hline
        \( \gls{constante:fixa:tempo-turno} \) &
        420                                    &
        minutos
        \\
        \hline
    \end{tabular}
    \fonte{\ComponenteFontePropria}
\end{quadro}

\subsubsubsection{Dados}

\begin{symbols}
    \item[\( \gls{constante:dado:receita:rendimento} \)]
    \glsentrydesc{constante:dado:receita:rendimento},
    tal que \( \glsentryname{constante:dado:receita:rendimento} \in \mathbb{R}^{+} \);

    \item[\( \gls{constante:dado:receita:rendimento-total} \)]
    \glsentrydesc{constante:dado:receita:rendimento-total},
    tal que \( \glsentryname{constante:dado:receita:rendimento-total} \in \mathbb{R}^{+} \);

    \item[\( \gls{constante:dado:receita:tempo-mistura} \)]
    \glsentrydesc{constante:dado:receita:tempo-mistura},
    tal que \( \glsentryname{constante:dado:receita:tempo-mistura} \in \mathbb{N}^{+} \);

    \item[\( \gls{constante:dado:pacote:gramatura} \)]
    \glsentrydesc{constante:dado:pacote:gramatura},
    tal que \( \glsentryname{constante:dado:pacote:gramatura} \in \mathbb{N}^{+} \);

    \item[\( \gls{constante:dado:pedido:peso} \)]
    \glsentrydesc{constante:dado:pedido:peso},
    tal que \( \glsentryname{constante:dado:pedido:peso} \in \mathbb{R}^{+} \).
\end{symbols}

\subsubsection{Variáveis}

\subsubsubsection{Auxiliares}

Todas as variáveis auxiliares acrescentam uma coluna à matriz de dados, sendo calculadas a partir de outras variáveis.

As variáveis auxiliares calculadas a partir do problema são:

\begin{symbols}
    \item[\( \gls{variavel:auxiliar:peso-total-pedidos-atendidos} \)]
    \glsentrydesc{variavel:auxiliar:peso-total-pedidos-atendidos},
    tal que \( \glsentryname{variavel:auxiliar:peso-total-pedidos-atendidos} \in \mathbb{R} \);

    \item[\( \gls{variavel:auxiliar:peso-total-produzido} \)]
    \glsentrydesc{variavel:auxiliar:peso-total-produzido},
    tal que \( \glsentryname{variavel:auxiliar:peso-total-produzido} \in \mathbb{R} \);

    \item[\( \gls{variavel:auxiliar:peso-total-em-sobra} \)]
    \glsentrydesc{variavel:auxiliar:peso-total-em-sobra},
    tal que \( \glsentryname{variavel:auxiliar:peso-total-em-sobra} \in \mathbb{R} \);
\end{symbols}

\subsubsubsection{Principais}

\begin{symbols}
    \item[\( \gls{variavel:principal:quantidade-porcoes-receita-a-produzir} \)]
    \glsentrydesc{variavel:principal:quantidade-porcoes-receita-a-produzir},
    tal que \( \glsentryname{variavel:principal:quantidade-porcoes-receita-a-produzir} \in \mathbb{N} \).
\end{symbols}

\subsection{Restrições}

\subsubsection{Auxiliares}

Todas as variáveis auxiliares incorrem em restrições de igualdade, uma vez que são calculadas segundo colunas da matriz de dados.

As restrições de igualdade que definem as variáveis auxiliares são:

\begin{align}
    \gls{variavel:auxiliar:peso-total-produzido} &
    = \gls{variavel:principal:quantidade-porcoes-receita-a-produzir} \times \gls{constante:dado:receita:rendimento}.
    \\
    \gls{variavel:auxiliar:peso-total-em-sobra}  &
    = \gls{variavel:auxiliar:peso-total-produzido} - \gls{variavel:auxiliar:peso-total-pedidos-atendidos}.
    \\
    \sum \gls{constante:dado:rendimento-total}   &
    \geq
    \sum \gls{constante:dado:pedido:peso}
    \\
    \gls{constante:dado:receita:tempo-mistura}   &
    \leq 420
\end{align}

Tempo máximo de mistura por expediente:

\begin{align}
    \gls{variavel:principal:quantidade-porcoes-receita-a-produzir}
    \times \gls{constante:dado:receita:tempo-mistura}
    \leq \gls{constante:fixa:tempo-turno}.
\end{align}

\subsection{Função-Objetivo}

A função objetivo deste problema é minimizar o peso total em sobra das receitas, conforme mostrado abaixo:

\begin{align}
    \min \sum \gls{variavel:auxiliar:peso-total-em-sobra}_{\gls{indice:pedidos}}
\end{align}


\section{Resultados}%
\label{sec:resultados}


\section{Conclusão}%
\label{sec:conclusao}

% =====

% =====
% ELEMENTOS PÓS-TEXTUAIS
% =====
\postextual{}

% --- Referências ---
\bibliography{cx_bibliografia}

% --- Glossário ---
\printglossary[type=main,style=altlist,title=Glossário]
\printglossary[type=acronym,style=altlist,title=Lista de Abreviaturas e Siglas]
% =====

\end{document}