\documentclass[
    % -- opções da classe memoir --
    article,			% indica que é um artigo acadêmico
    11pt,				% tamanho da fonte
    oneside,			% para impressão apenas no recto. Oposto a twoside
    a4paper,			% tamanho do papel. 
    % -- opções da classe abntex2 --
    %chapter=TITLE,		% títulos de capítulos convertidos em letras maiúsculas
    %section=TITLE,		% títulos de seções convertidos em letras maiúsculas
    %subsection=TITLE,	% títulos de subseções convertidos em letras maiúsculas
    %subsubsection=TITLE % títulos de subsubseções convertidos em letras maiúsculas
    % -- opções do pacote babel --
    english,			% idioma adicional para hifenização
    brazil,				% o último idioma é o principal do documento
    sumario=tradicional
]{abntex2}

% Pacotes
\input{packages.tex}

% Definições do trabalho
\input{definitions.tex}

% Configurações do documento
\input{configuration.tex}

% Início do documento
\begin{document}

% -----
% Configurações do texto
% -----
% Seleciona o idioma do documento
\selectlanguage{brazil}

% Retira espaço extra obsoleto entre as frases
\frenchspacing{}
% -----

% =====
% ELEMENTOS PRÉ-TEXTUAIS
% =====
\pretextual{}

% Página de titulo principal (obrigatório)
\maketitle{}

% Resumos
\input{abstract}

\begin{center}\smaller{}
    \textbf{Data de submissão e aprovação}: [INSIRA A DATA AQUI].
\end{center}
% =====

% =====
% ELEMENTOS TEXTUAIS
% =====
\textual{}

\section{Introdução}%
\label{sec:introducao}



\section{Metodologia}%
\label{sec:metodologia}

\begin{figure}[!ht]%
    \centering
    \includegraphics[scale=0.5]{img/black-square.png}
    \caption{Quadrado preto. Fonte: \citeonline{tortinhas}.}%
    \label{fig:f1}
\end{figure}



\input{results}

\input{conclusion}
% =====

% =====
% ELEMENTOS PÓS-TEXTUAIS
% =====
\postextual{}

% --- Referências ---
\bibliography{bibliography}

% --- Glossário ---
\printglossary[type=main,style=altlist,title=Glossário]
\printglossary[type=acronym,style=altlist,title=Lista de Abreviaturas e Siglas]
% =====

\end{document}