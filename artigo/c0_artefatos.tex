% =====
% Conteúdo dos artefatos
% =====

% -----
% Resumo (Obrigatório)
% -----

% --- Língua vernácula ---

% Palavras chave. Parâmetros: palavra-chave um; palavra-chave dois; palavra-chave três; palavra-chave quatro (opcional); palavra-chave cinco (opcional). Deixe em branco os campos opcionais que não deseja preencher.
\palavrasChave{palavra um}{palavra dois}{palavra três}{palavra quatro}{}

\insereResumo{%
    \lipsum[1-3]
}

% --- Língua estrangeira ---

% Parâmetros: língua (english, french, spanish, italian, german, dutch); palavras-chave; conteúdo.

\insereResumoEmLinguaEstrangeira%
{english}
{\formataPalavrasChave{word one}{word two}{word three}{word four}{}}
{%
    \lipsum[4-6]
}

\insereResumoEmLinguaEstrangeira%
{french}
{\formataPalavrasChave{word one}{word two}{word three}{word four}{}}
{%
    \lipsum[30-32]
}

% -----

% -----
% Agradecimentos (Opcional)
% -----

\insereAgradecimentos{%
    Agradecemos ao projeto \abnTeX\footnote{%
        Acesso em:~\url{http://www.abntex.net.br/}.%
    }, que disponibiliza o modelo \LaTeX\ que foi customizado para a elaboração de trabalhos acadêmicos conforme as normas da ABNT.

    Agradecemos também ao professor Dr.\ Jairo Souza, que desenvolveu o modelo \LaTeX\ para o \gls{dcc} da \gls{ufjf}, o qual baseou partes deste modelo.
}

% -----

% =====
