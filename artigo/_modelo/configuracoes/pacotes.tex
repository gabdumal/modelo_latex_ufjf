% =====
% Pacotes
% =====

\usepackage{lmodern}        % Usa a fonte Latin Modern.
\usepackage[T1]{fontenc}    % Seleção de códigos de fonte.
\usepackage[utf8]{inputenc} % Codificação do documento (conversão automática dos acentos).
\usepackage{indentfirst}    % Identa o primeiro parágrafo de cada seção.
\usepackage{microtype}      % Para melhorias de justificação.

\usepackage{color}              % Controle das cores.
\usepackage[dvipsnames]{xcolor} % Controle das cores (avançado).

\usepackage{graphicx}   % Inclusão de gráficos.

\usepackage{amssymb}    % Símbolos matemáticos.
\usepackage{amsmath}    % Equações matemáticas.
\usepackage{amsfonts}   % Fontes e símbolos matemáticos.

\usepackage[brazilian,hyperpageref]{backref}    % Páginas com as citações na bibliografia.
\usepackage{pdfpages}   % Inclusão de documentos PDF.

\usepackage[
    symbols,
    nohypertypes={symbols,index},
    % nonumberlist=true,
    seeautonumberlist,
    subentrycounter,
    toc=false
]{glossaries-extra}   % Glossário. Permite definir termos, siglas e abreviações. Deve ser carregado depois de backref.

\usepackage[alf]{abntex2cite}       % Citações padrão ABNT. Deve ser carregado depois de glossaries.

\usepackage{multicol}   % Permite incluir ambientes que exibem texto em múltiplas colunas.
\usepackage{multirow}   % Permite mesclar células em tabelas.
\usepackage{longtable}  % Permite criar tabelas com mais de uma página.

\usepackage{titlecaps}  % Fornece ferramentas para capitalização de palavras.
\usepackage{outlines}   % Permite criar listas com diferentes níveis de hierarquia.

\usepackage{xparse}         % Permite a definição de comandos com argumentos opcionais.
\usepackage{xstring}        % Fornece ferramentas para manipulação de strings.

\usepackage[portuguese]{todonotes}  % Adiciona notas e afazeres no documento.
\usepackage{lipsum}                 % Gera texto de preenchimento.

% =====

% =====
% Configurações dos pacotes
% =====

% -----
% abnTeX2
% -----

% Possibilita criação de Quadros e os insere na lista de ilustrações.
% Ver https://github.com/abntex/abntex2/issues/176

% Define o nome do tipo de ilustração.
\newcommand{\quadroname}{Quadro}

% Insere o quadro na lista de ilustrações.
\newfloat[section]{quadro}{lof}{\quadroname}
\newlistentry{quadro}{lof}{0}

% Define a formatação do quadro.
\setfloatadjustment{quadro}{\centering}
\counterwithout{quadro}{section}
\renewcommand{\cftquadroname}{\quadroname\space}
\renewcommand*{\cftquadroaftersnum}{\hfill--\hfill}
\setfloatlocations{quadro}{hbtp}

% -----

% -----
% Glossaries-extra
% -----

% O pacote "glossaries-extra" provê a opção de criar um glossário chamado "index" nas opções do pacote.
% Entretanto, isso cria um comando chamado "\printindex", que conflita com o homônimo do pacote "memoir", utilizado para imprimir o índice remissivo.
% Para resolver, não habilitamos a opção "index" e criamos um novo glossário chamado "index" manualmente.

\newglossary[ilg]{index}{ind}{idx}{\indexname}

% Define o comando "\newterm" para facilitar a criação de novos termos no glossário "index".
\DeclareDocumentCommand\newterm{m m g}{%
    \IfValueTF{#3}{%
        \newglossaryentry{#1}{%
            type=index,
            name={#2},
            description={\nopostdesc},
            plural={#3}%
        }
    }{%
        \newglossaryentry{#1}{%
            type=index,
            name={#2},
            description={\nopostdesc}%
        }
    }
}

% Permite a impressão dos glossários.
\makeglossaries{}
\loadglsentries{c0_glossario.tex}

% Evita quebra de página entre os glossários.
\renewcommand*{\glsclearpage}{}

% -----

% -----
% Backref
% -----

% Usado sem a opção hyperpageref de backref.
\renewcommand{\backrefpagesname}{Citado na(s) página(s):~} % chktex-file 36

% Texto padrão antes do número das páginas.
\renewcommand{\backref}{}

% Define os textos da citação.
\renewcommand*{\backrefalt}[4]{%
    \ifcase #1% chktex-file 1
        Nenhuma citação no texto.%
    \or%
        Citado na página #2.%
    \else%
        Citado #1 vezes nas páginas #2.%
    \fi}%

% -----

% -----
% Todonotes
% -----

% Define a largura da caixa de notas.
\setlength{\marginparwidth}{2cm}

% Define a cor e o estilo da caixa de notas.
\presetkeys{todonotes}{%
    inline,
    backgroundcolor=yellow,
    % disable % Comente para habilitar as notas.
}{}

% -----

% -----
% Titlecaps
% -----

% Define as palavras que não devem ser capitalizadas. Elas ainda serão capitalizadas se estiverem no início de uma frase.
\Addlcwords{a o e um uma uns umas em de da do das dos no na nos nas num numa nuns numas por para com pelo pela pelos pelas}

% -----

% =====
