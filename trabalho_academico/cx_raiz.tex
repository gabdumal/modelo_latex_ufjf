\documentclass[
    % --- Opções da classe memoir ---
	openright,      % Define que capítulos começam em página ímpar (insere página vazia caso preciso).
    12pt,           % Tamanho da fonte.
    %oneside,       % Para impressão apenas no anverso. Oposto a twoside.
    twoside,        % Para impressão em verso e anverso. Caso necessário, páginas do verso ficarão em branco. Oposto a oneside.
    a4paper,        % Tamanho do papel.
    % -----
    % --- Opções da classe abntex2 ---
    %chapter=TITLE,         % Títulos de capítulos são convertidos para letras maiúsculas.
    %section=TITLE,         % Títulos de seções são convertidos para letras maiúsculas.
    %subsection=TITLE,      % Títulos de subseções são convertidos para letras maiúsculas.
    %subsubsection=TITLE,   % Títulos de subsubseções são convertidos para letras maiúsculas.
    % -----
    % --- Opções do pacote babel ---
    english,			% Define idioma adicional para hifenização
    french,			    % Define idioma adicional para hifenização
    spanish,			% Define idioma adicional para hifenização
    brazil,				% O último idioma é definido como o principal do documento.
]{abntex2}

% --- Modelo ---
% =====
% Modelo
% =====

% --- Pacotes ---
% =====
% Pacotes
% =====

\usepackage{lmodern}        % Usa a fonte Latin Modern.
\usepackage[T1]{fontenc}    % Seleção de códigos de fonte.
\usepackage[utf8]{inputenc} % Codificação do documento (conversão automática dos acentos).
\usepackage{indentfirst}    % Identa o primeiro parágrafo de cada seção.
\usepackage{microtype}      % Para melhorias de justificação.

\usepackage{color}              % Controle das cores.
\usepackage[dvipsnames]{xcolor} % Controle das cores (avançado).

\usepackage{graphicx}   % Inclusão de gráficos.

\usepackage{amssymb}    % Símbolos matemáticos.
\usepackage{amsmath}    % Equações matemáticas.
\usepackage{amsfonts}   % Fontes e símbolos matemáticos.

\usepackage[brazilian,hyperpageref]{backref}    % Páginas com as citações na bibliografia.
\usepackage{pdfpages}   % Inclusão de documentos PDF.

\usepackage[
    symbols,
    nohypertypes={symbols,index},
    % nonumberlist=true,
    seeautonumberlist,
    subentrycounter,
    toc=false
]{glossaries-extra}   % Glossário. Permite definir termos, siglas e abreviações. Deve ser carregado depois de backref.

\usepackage[alf]{abntex2cite}       % Citações padrão ABNT. Deve ser carregado depois de glossaries.

\usepackage{multicol}   % Permite incluir ambientes que exibem texto em múltiplas colunas.
\usepackage{multirow}   % Permite mesclar células em tabelas.
\usepackage{longtable}  % Permite criar tabelas com mais de uma página.

\usepackage{titlecaps}  % Fornece ferramentas para capitalização de palavras.
\usepackage{outlines}   % Permite criar listas com diferentes níveis de hierarquia.

\usepackage{xparse}         % Permite a definição de comandos com argumentos opcionais.
\usepackage{xstring}        % Fornece ferramentas para manipulação de strings.

\usepackage[portuguese]{todonotes}  % Adiciona notas e afazeres no documento.
\usepackage{lipsum}                 % Gera texto de preenchimento.

% =====

% =====
% Configurações dos pacotes
% =====

% -----
% abnTeX2
% -----

% Possibilita criação de Quadros e os insere na lista de ilustrações.
% Ver https://github.com/abntex/abntex2/issues/176

% Define o nome do tipo de ilustração.
\newcommand{\quadroname}{Quadro}

% Insere o quadro na lista de ilustrações.
\newfloat[section]{quadro}{lof}{\quadroname}
\newlistentry{quadro}{lof}{0}

% Define a formatação do quadro.
\setfloatadjustment{quadro}{\centering}
\counterwithout{quadro}{section}
\renewcommand{\cftquadroname}{\quadroname\space}
\renewcommand*{\cftquadroaftersnum}{\hfill--\hfill}
\setfloatlocations{quadro}{hbtp}

% -----

% -----
% Glossaries-extra
% -----

% O pacote "glossaries-extra" provê a opção de criar um glossário chamado "index" nas opções do pacote.
% Entretanto, isso cria um comando chamado "\printindex", que conflita com o homônimo do pacote "memoir", utilizado para imprimir o índice remissivo.
% Para resolver, não habilitamos a opção "index" e criamos um novo glossário chamado "index" manualmente.

\newglossary[ilg]{index}{ind}{idx}{\indexname}

% Define o comando "\newterm" para facilitar a criação de novos termos no glossário "index".
\DeclareDocumentCommand\newterm{m m g}{%
    \IfValueTF{#3}{%
        \newglossaryentry{#1}{%
            type=index,
            name={#2},
            description={\nopostdesc},
            plural={#3}%
        }
    }{%
        \newglossaryentry{#1}{%
            type=index,
            name={#2},
            description={\nopostdesc}%
        }
    }
}

% Permite a impressão dos glossários.
\makeglossaries{}
\loadglsentries{c0_glossario.tex}

% Evita quebra de página entre os glossários.
\renewcommand*{\glsclearpage}{}

% -----

% -----
% Backref
% -----

% Usado sem a opção hyperpageref de backref.
\renewcommand{\backrefpagesname}{Citado na(s) página(s):~} % chktex-file 36

% Texto padrão antes do número das páginas.
\renewcommand{\backref}{}

% Define os textos da citação.
\renewcommand*{\backrefalt}[4]{%
    \ifcase #1% chktex-file 1
        Nenhuma citação no texto.%
    \or%
        Citado na página #2.%
    \else%
        Citado #1 vezes nas páginas #2.%
    \fi}%

% -----

% -----
% Todonotes
% -----

% Define a largura da caixa de notas.
\setlength{\marginparwidth}{2cm}

% Define a cor e o estilo da caixa de notas.
\presetkeys{todonotes}{%
    inline,
    backgroundcolor=yellow,
    % disable % Comente para habilitar as notas.
}{}

% -----

% -----
% Titlecaps
% -----

% Define as palavras que não devem ser capitalizadas. Elas ainda serão capitalizadas se estiverem no início de uma frase.
\Addlcwords{a o e um uma uns umas em de da do das dos no na nos nas num numa nuns numas por para com pelo pela pelos pelas}

% -----

% =====


% --- Comandos ---
% =====
% Comandos
% =====

% --- Variáveis ---
% =====
% Variáveis
% =====

% -----
% Flags e outras variáveis de decisão
% -----

% --- Título ---
\newif\ifDefinidoSubtitulo{} \DefinidoSubtitulofalse{}
\newif\ifDefinidoTituloEstrangeiro{} \DefinidoTituloEstrangeirofalse{}
\newif\ifDefinidoSubtituloEstrangeiro{} \DefinidoSubtituloEstrangeirofalse{}

% -----

% -----
% Variáveis de texto
% -----

% --- Título ---
\gdef\ValorDoTitulo{}
\def\DefineTitulo#1{\gdef\ValorDoTitulo{#1}}

\gdef\ValorDoSubtitulo{}
\def\DefineSubtitulo#1{\gdef\ValorDoSubtitulo{#1}\DefinidoSubtitulotrue}

\gdef\ValorDoTituloCompleto{%
    \ValorDoTitulo{}%
    \ifDefinidoSubtitulo{}: \ValorDoSubtitulo{}\fi%
}

\title{\ValorDoTituloCompleto}

\def\DefineTituloEstrangeiro#1{\gdef\ValorDoTituloEstrangeiro{#1}\DefinidoTituloEstrangeirotrue}

\def\DefineSubtituloEstrangeiro#1{\gdef\ValorDoSubtituloEstrangeiro{#1}\DefinidoSubtituloEstrangeirotrue}

\gdef\ValorDoTituloEstrangeiroCompleto{\ifDefinidoTituloEstrangeiro{}\ValorDoTituloEstrangeiro{}\ifDefinidoSubtituloEstrangeiro{}: \ValorDoSubtituloEstrangeiro{}\fi\fi}

\newcommand{\theforeigntitle}{\ValorDoTituloEstrangeiroCompleto}

% --- Trabalho ---

\def\DefineDataDeSubmissao#1#2#3{%
    \gdef\ValorDoDiaDeSubmissao{#1}
    \gdef\ValorDoMesDeSubmissao{#2}
    \gdef\ValorDoAnoDeSubmissao{#3}
}
\gdef\ValorDaDataDeSubmissao{\ValorDoMesDeSubmissao{}, \ValorDoAnoDeSubmissao{}}
\data{\ValorDaDataDeSubmissao}

\def\DefineDataDeAprovacao#1#2#3{%
    \gdef\ValorDoDiaDeAprovacao{#1}
    \gdef\ValorDoMesDeAprovacao{#2}
    \gdef\ValorDoAnoDeAprovacao{#3}
}
\gdef\ValorDaDataDeAprovacao{\ValorDoMesDeAprovacao{}, \ValorDoAnoDeAprovacao{}}
\data{\ValorDaDataDeAprovacao}

% --- Pessoas ---

\def\DefineAutor#1#2{\gdef\ValorDoUltimoSobrenomeDoAutor{#1} \gdef\ValorDoRestanteDoNomeDoAutor{#2} \gdef\ValorDoNomeCompletoDoAutor{#2 #1} \author{#2 #1}}

% --- Palavras-chave ---
\def\DefinePalavrasChave#1#2#3#4#5
{%
    \gdef\ValorDaPalavraChaveUm{#1}
    \gdef\ValorDaPalavraChaveDois{#2}
    \gdef\ValorDaPalavraChaveTres{#3}
    \gdef\ValorDaPalavraChaveQuatro{#4}
    \gdef\ValorDaPalavraChaveCinco{#5}
}

\NewDocumentCommand{\formataPalavrasChave}{m m m m m}
{%
    #1;
    #2;
    #3%
    \ifstrempty{#4}{}{; #4%
        \ifstrempty{#5}{}{; #5}%
    }%
}

\def\ValorDasPalavrasChave{
    \ValorDaPalavraChaveUm;
    \ValorDaPalavraChaveDois;
    \ValorDaPalavraChaveTres%
    \ifdefempty{\ValorDaPalavraChaveQuatro}{}{; \ValorDaPalavraChaveQuatro%
        \ifdefempty{\ValorDaPalavraChaveCinco}{}{; \ValorDaPalavraChaveCinco}%
    }%
}

% -----

\def\testaVariaveis{
    \ValorDoTitulo{}

    {%
        \ifDefinidoSubtitulo{%
            Subtítulo está definido:
            \ValorDoSubtitulo{}
        }
        \else{%
            Subtítulo não está definido.
        }
        \fi
    }

    {%

        \ifDefinidoTituloEstrangeiro{%
            Título estrangeiro está definido:
            \ValorDoTituloEstrangeiro{}

            {%
                \ifDefinidoSubtituloEstrangeiro{%
                    Subtítulo estrangeiro está definido:
                    \ValorDoSubtituloEstrangeiro{}
                }
                \else{%
                    Subtítulo estrangeiro não está definido.
                }
                \fi
            }

        }
        \else{%
            Título estrangeiro não está definido.
        }
        \fi
    }

    Data de submissão: \ValorDaDataDeSubmissao{}

    Data de aprovação: \ValorDaDataDeAprovacao{}

    \ValorDoUltimoSobrenomeDoAutor{} ---
    \ValorDoRestanteDoNomeDoAutor{} ---
    \ValorDoNomeCompletoDoAutor{}

    [\ValorDaPalavraChaveUm]
    [\ValorDaPalavraChaveDois]
    [\ValorDaPalavraChaveTres]
    [\ValorDaPalavraChaveQuatro]
    [\ValorDaPalavraChaveCinco]

    \ValorDasPalavrasChave.

    % \ifdefempty{\ValorDaPalavraChaveQuatro}{Quarta palavra-chave vazia.}{Quarta palavra-chave não vazia.}

    % \ifdefempty{\ValorDaPalavraChaveCinco}{Quinta palavra-chave vazia.}{Quinta palavra-chave não vazia.}

}

% =====


% --- Componentes ---
% =====
% Componentes
% =====

\gdef\ComponenteAutores{{%
            \flushright%
            \ABNTEXchapterfont%
            {\ValorDoNomeCompletoDoAutor{}}
            \par
        }}

\gdef\ComponenteTituloESubtitulo{{%
            \centering%
            \large%
            \textbf{\ValorDoTitulo{}}%
            \ifDefinidoSubtitulo%
                {\textbf{:} \ValorDoSubtitulo{}}%
            \fi
            \par
        }}

\gdef\ComponenteTituloESubtituloEstrangeiros{{%
            \ifDefinidoTituloEstrangeiro%
                \centering%
                \large%
                \textit{
                    \textbf{\ValorDoTituloEstrangeiro{}}%
                    \ifDefinidoSubtituloEstrangeiro%
                        {\textbf{:} \ValorDoSubtituloEstrangeiro{}}%
                    \fi
                }%
                \par
            \fi
        }}

\gdef\ComponenteDataEAcesso{{%
            \flushright%
            \ABNTEXchapterfont%
            Submetido em \ValorDaDataDeSubmissao{}.
            Aprovado em \ValorDaDataDeAprovacao{}.
            \\
            \ifDefinidoInformacoesDeAcesso%
                \ValorDasInformacoesDeAcesso{}
            \fi
            \par
        }}

\gdef\ComponenteFontePropria{
    Elaborado pelos autores (\ValorDoAnoDeSubmissao{}).
}

% =====


% =====


% --- Hifenização ---
\input{c0_hifenizacao.tex}

% --- Definições ---
% =====
% Definições para gerar os artefatos
% =====

% -----
% Título 
% -----

% Título principal do trabalho. Parâmetros: título.
\DefineTitulo{Título do meu trabalho}

% Subtítulo do trabalho. Parâmetros: subtítulo (opcional).
\DefineSubtitulo{O subtítulo do meu trabalho}

% Título principal do trabalho em língua estrangeira. Parâmetros: título.
\DefineTituloEstrangeiro{My work title}

% Subtítulo do trabalho em língua estrangeira. Parâmetros: subtítulo (opcional).
\DefineSubtituloEstrangeiro{The subtitle of my work}

% -----

% -----
% Trabalho
% -----

% Data de submissão. Parâmetros: dia; mês por extenso, em letras minúsculas; ano no formato 1000.
\DefineDataDeSubmissao{01}{janeiro}{2025}

% Data de aprovação. Parâmetros: dia; mês por extenso, em letras minúsculas; ano no formato 1000.
\DefineDataDeAprovacao{01}{março}{2025}

% -----

% -----

% -----
% Pessoas 
% -----

% Autor. Parâmetros: Último sobrenome; restante do nome.
\DefineAutor{Alves}{Alice Carvalho de}

% -----

% =====


% --- Artefatos ---
% =====
% Conteúdo dos artefatos
% =====

% -----
% Folha de rosto
% -----


% -----

% -----
% Ficha catalográfica
% -----

% Para a versão final do trabalho, sua instituição pode lhe fornecer um PDF com a versão definitiva da ficha catalográfica.

% Se for este o caso, salve o PDF no diretório "documentos" do seu projeto, então utilize a segunda versão do comando abaixo.

% Substitua "documentos/black-square.pdf" pelo caminho do PDF que você salvou no seu projeto.

\inserirFichaCatalografica%
% \inserirFichaCatalografica[%
%     \includepdf{documentos/black-square.pdf}
% ]

% -----

% -----
% Resumo
% -----

% --- Língua vernácula ---

% Palavras chave. Parâmetros: palavra-chave um; palavra-chave dois; palavra-chave três; palavra-chave quatro (opcional); palavra-chave cinco (opcional). Deixe em branco os campos opcionais que não deseja preencher.
\palavrasChave{palavra um}{palavra dois}{palavra três}{palavra quatro}{}

\conteudoDoResumo{%
    \lipsum[1-3]
}

% --- Língua estrangeira ---

% Parâmetros: língua (english, french, spanish, italian, german, dutch); palavras-chave; conteúdo.

\insereResumoEmLinguaEstrangeira%
{english}
{\formataPalavrasChave{word one}{word two}{word three}{word four}{}}
{%
    \lipsum[4-6]
}

\insereResumoEmLinguaEstrangeira%
{french}
{\formataPalavrasChave{word one}{word two}{word three}{word four}{}}
{%
    \lipsum[30-32]
}

% -----

% =====


% --- Configurações da aparência ---
% =====
% Configurações de aparência do PDF final
% =====

% Define o tom da cor azul.
\definecolor{blue}{RGB}{41,5,195}

% --- Informações do PDF ---
\hypersetup{%
    %pagebackref=true,
    pdftitle={\ValorDoTitulo},
    pdfauthor={\ValorDoNomeCompletoDoAutor},
    pdfsubject={Trabalho acadêmico:  \ValorDoTipo{} (\ValorDoDiploma{})},
    pdfcreator={LaTeX},
    pdfkeywords={\ValorDasPalavrasChave},
    % Estilo dos links (true: destacados por cor, false: destacados com uma caixa ao redor).
    colorlinks=true,
    % Cor dos links internos.
    linkcolor=black,
    % Cor dos links para a bibliografia.
    citecolor=black,
    % Cor dos links para arquivos.
    filecolor=black,
    % Cor dos links para URLs.
    urlcolor=black,
    bookmarksdepth=4
}

% =====


% --- Demais configurações ---
% -----
% Configurações dos pacotes
% -----

% --- ABNTEX2 ---
% Possibilita criação de Quadros e os insere na lista de ilustrações.
% Ver https://github.com/abntex/abntex2/issues/176

% Define o nome do tipo de ilustração.
\newcommand{\quadroname}{Quadro}
% Insere o quadro na lista de ilustrações.
\newfloat[chapter]{quadro}{lof}{\quadroname}
\newlistentry{quadro}{lof}{0}

% Define a formatação do quadro.
\setfloatadjustment{quadro}{\centering}
\counterwithout{quadro}{chapter}
\renewcommand{\cftquadroname}{\quadroname\space}
\renewcommand*{\cftquadroaftersnum}{\hfill--\hfill}
\setfloatlocations{quadro}{hbtp}

% --- Glossaries-extra ---
% O pacote "glossaries-extra" provê a opção de criar um glossário chamado "index" nas opções do pacote. Entretanto, isso cria um comando chamado "\printindex", que conflita com o homônimo do pacote "memoir", utilizado para imprimir o índice remissivo. Para resolver, não habilitamos a opção "index" e criamos um novo glossário chamado "index" manualmente.
\newglossary[ilg]{index}{ind}{idx}{\indexname}

% Define o comando "\newterm" para facilitar a criação de novos termos no glossário "index".
\DeclareDocumentCommand\newterm{m m g}{%
    \IfValueTF{#3}{%
        \newglossaryentry{#1}{%
            type=index,
            name={#2},
            description={\nopostdesc},
            plural={#3}%
        }
    }{%
        \newglossaryentry{#1}{%
            type=index,
            name={#2},
            description={\nopostdesc}%
        }
    }
}

% Permite a impressão dos glossários.
\makeglossaries{}
\loadglsentries{c0_glossario.tex}

% Evita quebra de página entre os glossários.
\renewcommand*{\glsclearpage}{}

% --- Backref ---
% Usado sem a opção hyperpageref de backref.
\renewcommand{\backrefpagesname}{Citado na(s) página(s):~} % chktex-file 36

% Texto padrão antes do número das páginas.
\renewcommand{\backref}{}

% Define os textos da citação.
\renewcommand*{\backrefalt}[4]{%
    \ifcase #1% chktex-file 1
        Nenhuma citação no texto.%
    \or%
        Citado na página #2.%
    \else%
        Citado #1 vezes nas páginas #2.%
    \fi}%

% --- Todonotes ---
% Define a largura da caixa de notas.
\setlength{\marginparwidth}{2cm}

% Define a cor e o estilo da caixa de notas.
\presetkeys{todonotes}{%
    inline,
    backgroundcolor=yellow,
    disable=false % Defina como "true" para desabilitar as notas.
}{}

% --- Titlecaps ---
% Define as palavras que não devem ser capitalizadas. Elas ainda serão capitalizadas se estiverem no início de uma frase.
\Addlcwords{a o e um uma uns umas em de da do das dos no na nos nas num numa nuns numas por para com pelo pela pelos pelas}

% -----


% =====


% Inicia o documento.
\begin{document}

% -----
% Configurações do texto
% -----
% Seleciona o idioma do documento.
\selectlanguage{brazil}
% Define a mesma largura para os espaços entre palavras de uma mesma sentença e entre sentenças diferentes. 
\frenchspacing{}
% -----

% =====
% ELEMENTOS PRÉ-TEXTUAIS
% =====
% =====
% Artefatos pré-textuais
% =====

\pretextual{}

% --- Capa ---
\imprimircapa{}

% --- Folha de rosto ---
% --- Ficha catalográfica ---
\ifDeveGerarFichaCatalografica%
    % Imprime a folha de rosto com a ficha catalográfica no verso.
    \imprimirfolhaderosto*
    \par
    \geraFichaCatalografica%
\else%
    % Imprime a folha de rosto com a ficha catalográfica em outra página.
    \imprimirfolhaderosto%
\fi

% --- Errata ---
\ifDefinidoConteudoDaErrata%
    \begin{errata}
        \ValorDoConteudoDaErrata{}
    \end{errata}
\fi

% --- Folha de aprovação ---
\ifDeveGerarFolhaDeAprovacao%
    \geraFolhaDeAprovacao%
\fi

% --- Dedicatória ---
\ifDefinidoConteudoDaDedicatoria%
    \begin{AmbienteDedicatoria}
        \ValorDoConteudoDaDedicatoria{}
    \end{AmbienteDedicatoria}
\fi

% --- Agradecimentos ---
\ifDefinidoConteudoDosAgradecimentos%
    \begin{AmbienteAgradecimentos}
        \ValorDoConteudoDosAgradecimentos{}
    \end{AmbienteAgradecimentos}
\fi

% --- Epígrafe ---
\ifDefinidoConteudoDaEpigrafe%
    \begin{AmbienteEpigrafe}
        \ValorDoConteudoDaEpigrafe{}
    \end{AmbienteEpigrafe}
\fi

% --- Resumos ---
\begin{AmbienteResumo}
    \ValorDoConteudoDoResumo{}
\end{AmbienteResumo}
\ValorDosResumosEmLinguasEstrangeiras{}

% --- Lista de ilustrações ---
\pdfbookmark[0]{\listfigurename}{lof}
\listoffigures*
\cleardoublepage{}

% --- Lista de tabelas ---
\pdfbookmark[0]{\listtablename}{lot}
\listoftables*
\cleardoublepage{}

% --- Lista de abreviaturas e siglas ---
\printglossary[type=abbreviations,style=altlist,title=\listadesiglasname]
\glsxtrifemptyglossary{abbreviations}{}{\cleardoublepage}

% --- Lista de símbolos ---
\printglossary[type=symbols,title=\listadesimbolosname]
\glsxtrifemptyglossary{symbols}{}{\cleardoublepage}

% --- Sumário ---
\pdfbookmark[0]{\contentsname}{toc}
\tableofcontents*
\cleardoublepage{}

% =====

% =====

% =====
% ELEMENTOS TEXTUAIS
% =====
\textual{}

\chapter{Introdução}%
\label{cap:introducao}

Este modelo se baseia no Modelo Canônico para trabalhos Acadêmicos do projeto \abnTeX~\cite{abntex2:2024}.

Também se utilizam recursos desenvolvidos pelo projeto de \citeonline{souza:2024}, que adapta as regras de formatação da \gls{ufjf} para um modelo em \LaTeX.

Por fim, as definições foram ajustadas conforme o Manual de normalização e modelos de trabalhos acadêmicos da \gls{ufjf}~\cite{cdd:2023}.


\chapter{Fundamentação Teórica}%
\label{cap:fundamentacao}

Apresentamos exemplos de uso de símbolos matemáticos e equações.

\section{Dados}

\subsection{Índices}

O problema permite delimitar os seguintes índices:

\begin{symbols}
    \item[\( \gls{indice:receitas} \)]
    \glsentrydesc{indice:receitas},
    tal que \(  \gls{indice:receitas} \in [1, N] \);

    \item[\( \gls{indice:pacotes} \)]
    \glsentrydesc{indice:pacotes},
    tal que \(  \gls{indice:pacotes} \in [1, N] \);

    \item[\( \gls{indice:pedidos} \)]
    \glsentrydesc{indice:pedidos},
    tal que \(  \gls{indice:pedidos} \in [1, N] \);
\end{symbols}

\subsection{Constantes}

\subsubsection{Fixas}

As seguinte constante é fixa e não pode ser alterada durante a execução do modelo.

\begin{symbols}
    \item[\( \gls{constante:fixa:tempo-turno} \) ]
    \glsentrydesc{constante:fixa:tempo-turno},
    tal que \(  \gls{constante:fixa:tempo-turno} \in \mathbb{N}^{+} \).
\end{symbols}

O valor da constante citada e sua unidade são exibidos no \autoref{qua:constantes-fixas}.

\begin{quadro}
    \caption{%
        \label{qua:constantes-fixas}%
        Constantes do problema.
    }

    \begin{tabular}{|c|c|c|}
        \hline
        Constante                              &
        Valor                                  &
        Unidade
        \\
        \hline
        \( \gls{constante:fixa:tempo-turno} \) &
        420                                    &
        minutos
        \\
        \hline
    \end{tabular}

    \ComponenteFontePropria{}
\end{quadro}

\subsubsection{Dados}

\begin{symbols}
    \item[\( \gls{constante:dado:receita:rendimento} \)]
    \glsentrydesc{constante:dado:receita:rendimento},
    tal que \( \glsentryname{constante:dado:receita:rendimento} \in \mathbb{R}^{+} \);

    \item[\( \gls{constante:dado:receita:rendimento-total} \)]
    \glsentrydesc{constante:dado:receita:rendimento-total},
    tal que \( \glsentryname{constante:dado:receita:rendimento-total} \in \mathbb{R}^{+} \);

    \item[\( \gls{constante:dado:receita:tempo-mistura} \)]
    \glsentrydesc{constante:dado:receita:tempo-mistura},
    tal que \( \glsentryname{constante:dado:receita:tempo-mistura} \in \mathbb{N}^{+} \);

    \item[\( \gls{constante:dado:pacote:gramatura} \)]
    \glsentrydesc{constante:dado:pacote:gramatura},
    tal que \( \glsentryname{constante:dado:pacote:gramatura} \in \mathbb{N}^{+} \);

    \item[\( \gls{constante:dado:pedido:peso} \)]
    \glsentrydesc{constante:dado:pedido:peso},
    tal que \( \glsentryname{constante:dado:pedido:peso} \in \mathbb{R}^{+} \).
\end{symbols}

\subsection{Variáveis}

\subsubsection{Auxiliares}

Todas as variáveis auxiliares acrescentam uma coluna à matriz de dados, sendo calculadas a partir de outras variáveis.

As variáveis auxiliares calculadas a partir do problema são:

\begin{symbols}
    \item[\( \gls{variavel:auxiliar:peso-total-pedidos-atendidos} \)]
    \glsentrydesc{variavel:auxiliar:peso-total-pedidos-atendidos},
    tal que \( \glsentryname{variavel:auxiliar:peso-total-pedidos-atendidos} \in \mathbb{R} \);

    \item[\( \gls{variavel:auxiliar:peso-total-produzido} \)]
    \glsentrydesc{variavel:auxiliar:peso-total-produzido},
    tal que \( \glsentryname{variavel:auxiliar:peso-total-produzido} \in \mathbb{R} \);

    \item[\( \gls{variavel:auxiliar:peso-total-em-sobra} \)]
    \glsentrydesc{variavel:auxiliar:peso-total-em-sobra},
    tal que \( \glsentryname{variavel:auxiliar:peso-total-em-sobra} \in \mathbb{R} \);
\end{symbols}

\subsubsection{Principais}

\begin{symbols}
    \item[\( \gls{variavel:principal:quantidade-porcoes-receita-a-produzir} \)]
    \glsentrydesc{variavel:principal:quantidade-porcoes-receita-a-produzir},
    tal que \(  \glsentryname{variavel:principal:quantidade-porcoes-receita-a-produzir} \in \mathbb{N} \).
\end{symbols}

\section{Restrições}

\subsection{Auxiliares}

Todas as variáveis auxiliares incorrem em restrições de igualdade, uma vez que são calculadas segundo colunas da matriz de dados.

As restrições de igualdade que definem as variáveis auxiliares são:

\begin{align}
    \gls{variavel:auxiliar:peso-total-produzido} &
    = \gls{variavel:principal:quantidade-porcoes-receita-a-produzir} \times \gls{constante:dado:receita:rendimento}.
    \\
    \gls{variavel:auxiliar:peso-total-em-sobra}  &
    = \gls{variavel:auxiliar:peso-total-produzido} - \gls{variavel:auxiliar:peso-total-pedidos-atendidos}.
    \\
    \sum \gls{constante:dado:rendimento-total}   &
    \geq
    \sum \gls{constante:dado:pedido:peso}
    \\
    \gls{constante:dado:receita:tempo-mistura}   &
    \leq 420
\end{align}

Tempo máximo de mistura por expediente:

\begin{align}
    \gls{variavel:principal:quantidade-porcoes-receita-a-produzir}
    \times \gls{constante:dado:receita:tempo-mistura}
    \leq \gls{constante:fixa:tempo-turno}.
\end{align}

\section{Função-Objetivo}

A função objetivo deste problema é minimizar o peso total em sobra das receitas, conforme mostrado abaixo:

\begin{align}
    \min \sum \gls{variavel:auxiliar:peso-total-em-sobra}_{\gls{indice:pedidos}}
\end{align}


\section{Trabalhos Relacionados}%
\label{sec:relacionados}

\todo{Afazer}

Este trabalho foi realizado no \gls{dcc}, da unidade acadêmica \gls{ice}, da \gls{ufjf}.
Seu objetivo é tratar sobre \glspl{ga}, como introduzidos pela disciplina \gls{disciplina}.
Abordamos os temas de \gls{fitness}, \gls{crossover}, \gls{conjunto_vazio}, \glspl{fn}.


A \autoref{tab:classes_de_equivalencia_por_particionamento} apresenta as classes de equivalência obtidas por meio de particionamento das condições de entrada.

\begin{table}[htb]
    \IBGEtab{%
        \caption{Classes de equivalência obtidas por meio de particionamento das condições de entrada}%
        \label{tab:classes_de_equivalencia_por_particionamento}
    }{%
        \begin{tabular}{p{4.25cm}p{4.25cm}p{4.25cm}}
            \toprule
            \textbf{Condição de entrada}
             &
            \textbf{Classes de equivalência válidas}
             &
            \textbf{Classes de equivalência inválidas}
            \\

            \midrule

            \multirow{2}{4.25cm}{Estado atual da célula (\textcolor{Blue}{E})}
             &
            O estado atual da célula é vivo (\textcolor{Green}{V1})
             &
            \
            \\

            \
             &
            O estado atual da célula é morto (\textcolor{Green}{V2})
             &
            \
            \\

            \midrule

            \multirow{2}{4.25cm}{Quantidade de vizinhos vivos (\textcolor{Blue}{V})}
             &
            \multirow{2}{4.25cm}{Está no intervalo \( 0 \leq \mathcolor{Blue}{V} \leq 8 \) (\textcolor{Green}{V3})}
             &
            Abaixo do limite inferior: \( \mathcolor{Blue}{V} < 0 \) (\textcolor{Red}{I1})
            \\

            \
             &
            \
             &
            Acima do limite superior: \( \mathcolor{Blue}{V} > 8 \) (\textcolor{Red}{I2})
            \\

            \bottomrule
        \end{tabular}%
    }{%
        \fonte{\ComponenteFontePropria}%
    }
\end{table}

\subsection{Teste das variáveis}%
\label{sec:teste_variaveis}

\testaVariaveis{}


\chapter{Resultados}%
\label{cap:resultados}%
\index{Índice para os resultados}

\begin{figure}%
    \caption{\label{fig:f1}Quadrado preto}%
    \centering
    \includegraphics[scale=0.5]{imagens/black-square.png}
    \legend{Fonte:~\citeonline{tortinhas:2024}}
\end{figure}

\lipsum[1-40] % chktex-file 8


\section{Conclusão}%
\label{cap:conclusao}

% =====

% =====
% ELEMENTOS PÓS-TEXTUAIS
% =====
% =====
% Artefatos pré-textuais
% =====

\pretextual{}

% --- Capa ---
\imprimircapa{}

% --- Folha de rosto ---
% --- Ficha catalográfica ---
\ifDeveGerarFichaCatalografica%
    % Imprime a folha de rosto com a ficha catalográfica no verso.
    \imprimirfolhaderosto*
    \par
    \geraFichaCatalografica%
\else%
    % Imprime a folha de rosto com a ficha catalográfica em outra página.
    \imprimirfolhaderosto%
\fi

% --- Errata ---
\ifDefinidoConteudoDaErrata%
    \begin{errata}
        \ValorDoConteudoDaErrata{}
    \end{errata}
\fi

% --- Folha de aprovação ---
\ifDeveGerarFolhaDeAprovacao%
    \geraFolhaDeAprovacao%
\fi

% --- Dedicatória ---
\ifDefinidoConteudoDaDedicatoria%
    \begin{AmbienteDedicatoria}
        \ValorDoConteudoDaDedicatoria{}
    \end{AmbienteDedicatoria}
\fi

% --- Agradecimentos ---
\ifDefinidoConteudoDosAgradecimentos%
    \begin{AmbienteAgradecimentos}
        \ValorDoConteudoDosAgradecimentos{}
    \end{AmbienteAgradecimentos}
\fi

% --- Epígrafe ---
\ifDefinidoConteudoDaEpigrafe%
    \begin{AmbienteEpigrafe}
        \ValorDoConteudoDaEpigrafe{}
    \end{AmbienteEpigrafe}
\fi

% --- Resumos ---
\begin{AmbienteResumo}
    \ValorDoConteudoDoResumo{}
\end{AmbienteResumo}
\ValorDosResumosEmLinguasEstrangeiras{}

% --- Lista de ilustrações ---
\pdfbookmark[0]{\listfigurename}{lof}
\listoffigures*
\cleardoublepage{}

% --- Lista de tabelas ---
\pdfbookmark[0]{\listtablename}{lot}
\listoftables*
\cleardoublepage{}

% --- Lista de abreviaturas e siglas ---
\printglossary[type=abbreviations,style=altlist,title=\listadesiglasname]
\glsxtrifemptyglossary{abbreviations}{}{\cleardoublepage}

% --- Lista de símbolos ---
\printglossary[type=symbols,title=\listadesimbolosname]
\glsxtrifemptyglossary{symbols}{}{\cleardoublepage}

% --- Sumário ---
\pdfbookmark[0]{\contentsname}{toc}
\tableofcontents*
\cleardoublepage{}

% =====

% =====

\end{document}
