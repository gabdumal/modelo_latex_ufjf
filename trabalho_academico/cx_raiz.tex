\documentclass[
    % --- Opções da classe memoir ---
	openright,			% Define que capítulos começam em página ímpar (insere página vazia caso preciso).
    12pt,				% Tamanho da fonte.
    %oneside,			% Para impressão apenas no anverso. Oposto a twoside.
    twoside,             % Para impressão em verso e anverso. Caso necessário, páginas do verso ficarão em branco. Oposto a oneside.
    a4paper,			% tamanho do papel.
    % -----
    % --- Opções da classe abntex2 ---
    %chapter=TITLE,		% Títulos de capítulos são convertidos para letras maiúsculas.
    %section=TITLE,		% Títulos de seções são convertidos para letras maiúsculas.
    %subsection=TITLE,	% Títulos de subseções são convertidos para letras maiúsculas.
    %subsubsection=TITLE, % Títulos de subsubseções são convertidos para letras maiúsculas.
    % -----
    % --- Opções do pacote babel ---
    english,			% Define idioma adicional para hifenização
    french,			    % Define idioma adicional para hifenização
    spanish,			% Define idioma adicional para hifenização
    brazil,				% O último idioma é definido como o principal do documento.
]{abntex2}

% --- Pacotes ---
% -----
% Pacotes fundamentais 
% -----

\usepackage{lmodern}        % Usa a fonte Latin Modern.
\usepackage[T1]{fontenc}    % Seleção de códigos de fonte.
\usepackage[utf8]{inputenc} % Codificação do documento (conversão automática dos acentos).
\usepackage{indentfirst}    % Identa o primeiro parágrafo de cada seção.
\usepackage{microtype}      % Para melhorias de justificação.

\usepackage{color}      % Controle das cores.
\usepackage{graphicx}   % Inclusão de gráficos.

\usepackage[brazilian,hyperpageref]{backref}    % Páginas com as citações na bibliografia.

\usepackage[subentrycounter,seeautonumberlist,nonumberlist=true,acronym,nohypertypes={index}]{glossaries}   % Glossário. Permite definir termos, siglas e abreviações. Deve ser carregado depois de backref.

\usepackage[alf]{abntex2cite}       % Citações padrão ABNT. Deve ser carregado depois de glossaries.

\usepackage[portuguese]{todonotes}  % Adiciona notas e afazeres no documento.

\usepackage{multirow}   % Permite mesclar células em tabelas.
\usepackage{longtable}  % Permite criar tabelas com mais de uma página.

\usepackage{lipsum}     % Gera texto de preenchimento.

% -----

% -----
% Configurações dos pacotes
% -----

% --- Glossaries ---
\newglossary[ilg]{index}{ind}{idx}{\indexname}
\newcommand*{\newterm}[2]{
    \newglossaryentry{#1}
    {type=index,name={#2},description={\nopostdesc}}
}
\makeglossaries{} % Habilite este comando para permitir a impressão dos glossários.
\loadglsentries{c0_glossario.tex}
\renewcommand*{\glsclearpage}{} % Evita quebra de página entre os glossários.

% --- Backref ---
% Usado sem a opção hyperpageref de backref.
\renewcommand{\backrefpagesname}{Citado na(s) página(s):~}
% Texto padrão antes do número das páginas.
\renewcommand{\backref}{}
% Define os textos da citação.
\renewcommand*{\backrefalt}[4]{%
    \ifcase #1%
        Nenhuma citação no texto.%
    \or%
        Citado na página #2.%
    \else%
        Citado #1 vezes nas páginas #2.%
    \fi}%

% --- Todonotes ---
% Define a largura da caixa de notas.
\setlength{\marginparwidth}{2cm}
% Define a cor e o estilo da caixa de notas.
\presetkeys{todonotes}{inline,backgroundcolor=yellow}{}
% Desabilita as notas.
% \presetkeys{todonotes}{disable}{}

% -----

% --- Comandos ---
% --- Flags ---
\newif\ifdefinidoSubtitulo{} \definidoSubtitulofalse{}

% --- Variáveis ---
\def\subtitulo#1{\gdef\valorDoSubtitulo{#1}\definidoSubtitulotrue}

\def\tipoDeTrabalho#1{\gdef\valorDoTipoDeTrabalho{#1}}


% --- Definições pre-textuais ---
% -----
% Informações de dados para gerar os ARTEFATOS
% -----

% --- Título ---
% Título principal do trabalho. Parâmetros: título.
\titulo{Título do meu trabalho}
% Subtítulo do trabalho. Parâmetros: subtítulo (opcional).
\subtitulo{O subtítulo do meu trabalho}

% --- Trabalho ---
% Tipo de trabalho. Parâmetros: classificação (doutorado, mestrado, especializacao, licenciatura, bacharelado).
\tipo{bacharelado}

% Curso. Parâmetros: nome do curso.
\curso{Sistemas de Informação}

% -- Data da aprovação --
% Dia. Parâmetros: dia no formato 01.
\dia{01}
% Mês. Parâmetros: mês por extenso, em letras minúsculas.
\mes{janeiro}
% Ano. Parâmetros: ano no formato 1000.
\ano{2025}

% -- Local de publicação --
% Cidade. Parâmetros: nome da cidade.
\cidade{Juiz de Fora}
% Estado. Parâmetros: nome do estado.
\estado{Minas Gerais}
% País. Parâmetros: nome do país.
\pais{Brasil}

% --- Instituição ---
% Instituição. Parâmetros: nome da instituição; sigla da instituição; cidade da instituição.
\instituicao{Universidade Federal de Juiz de Fora}{UFJF}{Juiz de Fora}
% Unidade acadêmica. Parâmetros: nome da unidade acadêmica; sigla da unidade acadêmica.
\unidadeAcademica{Instituto de Ciências Exatas}{ICE}
% Departamento. Parâmetros: nome do departamento; sigla do departamento.
\departamento{Departamento de Ciência da Computação}{DCC}

% --- Pessoas ---
% Autor. Parâmetros: Último sobrenome; restante do nome.
\autor{Alves}{Alice Carvalho de}

% Orientador. Parâmetros: título; último sobrenome; restante do nome; instituição de afiliação.
\orientador{Profa.\ Dra.}{Batista}{Beatriz Alves}{Universidade Federal de Juiz de Fora}

% Coorientador. Parâmetros: título; último sobrenome; restante do nome; instituição de afiliação.
\coorientador{Prof.\ Dr.}{Carvalho}{Carlos Ribeiro de}{Universidade Federal de Juiz de Fora}

% Examinador um. Parâmetros: título; último sobrenome; restante do nome; instituição de afiliação.
\examinadorUm{Prof.\ Dr.}{Dias}{Daniel Pereira}{Universidade Federal de Juiz de Fora}

% Examinador dois. Parâmetros: título; último sobrenome; restante do nome; instituição de afiliação.
\examinadorDois{Profa.\ Dra.}{Espinoza}{Eva Maria}{Universidade Federal de Juiz de Fora}

% Examinador três. Parâmetros: título; último sobrenome; restante do nome; instituição de afiliação.
\examinadorTres{Prof.\ Dr.}{Ferreira}{Fernando dos Santos}{Universidade Federal de Juiz de Fora}

% Examinador quatro. Parâmetros: título; último sobrenome; restante do nome; instituição de afiliação.
\examinadorQuatro{Profa.\ Dra.}{Gonçalves}{Gabriela Almeida}{Universidade Federal de Juiz de Fora}

% --- Resumo ---
% Palavras chave. Parâmetros: palavra-chave um; palavra-chave dois; palavra-chave três; palavra-chave quatro (opcional); palavra-chave cinco (opcional). Os parâmetros opcionais devem ser envolvidos por colchetes. Todas as palavras devem estar em letras minúsculas, salvo para siglas.
\palavrasChave{palavra um}{palavra dois}{palavra três}[palavra quatro]

% -----


% --- Configurações do documento ---
% -----
% Configurações de aparência do PDF final
% -----
\definecolor{blue}{RGB}{41,5,195}
% Informações do PDF
\makeatletter
\hypersetup{%
    %pagebackref=true,
    pdftitle={\@title},
    pdfauthor={\@author},
    pdfsubject={Modelo de artigo científico com abnTeX2},
    pdfcreator={LaTeX with abnTeX2},
    pdfkeywords={abnt}{latex}{abntex}{abntex2}{artigo científico},
    colorlinks=true,       		% false: boxed links; true: colored links
    linkcolor=blue,          	% color of internal links
    citecolor=blue,        		% color of links to bibliography
    filecolor=magenta,      		% color of file links
    urlcolor=blue,
    bookmarksdepth=4
}
\makeatother
% -----

% -----
% Demais configurações
% -----

% Compila o índice
\makeindex

% Altera as margens padrões
\setlrmarginsandblock{3cm}{3cm}{*}
\setulmarginsandblock{3cm}{3cm}{*}
\checkandfixthelayout{}

% Espaçamentos entre linhas e parágrafos
\setlength{\parindent}{1.3cm} % Tamanho do parágrafo
\setlength{\parskip}{0.2cm}  % Controle do espaçamento entre um parágrafo e outro
\SingleSpacing{} % Espaçamento simples

% -----

% Inicia o documento.
\begin{document}

% -----
% Configurações do texto
% -----
% Seleciona o idioma do documento.
\selectlanguage{brazil}

% Retira espaço extra obsoleto entre as frases.
\frenchspacing{}
% -----

% =====
% ELEMENTOS PRÉ-TEXTUAIS
% =====
\pretextual{}

% --- Capa ---
\imprimircapa{}

% --- Folha de rosto ---
% Imprime a folha de rosto com a ficha catalográfica.
\imprimirfolhaderosto*
% Imprime a folha de rosto sem a ficha catalográfica.
% \imprimirfolhaderosto{}

% --- Resumos ---
% --- Português ---
\begin{resumoumacoluna}

    % Escreva aqui o resumo em português
    Resumo.

    \testaVariaveis{}

    \vspace{\onelineskip}

    \noindent
    \textbf{Palavras-chave}: % Escreva aqui as palavras-chave em português
\end{resumoumacoluna}

% --- Língua estrangeira ---
\renewcommand{\resumoname}{Abstract}
\begin{resumoumacoluna}
    \begin{otherlanguage*}{english}

        % Write here the abstract in English
        Abstract.
        \vspace{\onelineskip}

        \noindent
        \textbf{Keywords}: % Write here the keywords in English
    \end{otherlanguage*}
\end{resumoumacoluna}

% =====

% =====
% ELEMENTOS TEXTUAIS
% =====
\textual{}

\chapter{Introdução}%
\label{cap:introducao}

Este modelo se baseia no Modelo Canônico para trabalhos Acadêmicos do projeto \abnTeX~\cite{abntex2:2024}.

Também se utilizam recursos desenvolvidos pelo projeto de \citeonline{souza:2024}, que adapta as regras de formatação da \gls{ufjf} para um modelo em \LaTeX.

Por fim, as definições foram ajustadas conforme o Manual de normalização e modelos de trabalhos acadêmicos da \gls{ufjf}~\cite{cdd:2023}.


\chapter{Fundamentação Teórica}%
\label{cap:fundamentacao}

Apresentamos exemplos de uso de símbolos matemáticos e equações.

\section{Dados}

\subsection{Índices}

O problema permite delimitar os seguintes índices:

\begin{symbols}
    \item[\( \gls{indice:receitas} \)]
    \glsentrydesc{indice:receitas},
    tal que \(  \gls{indice:receitas} \in [1, N] \);

    \item[\( \gls{indice:pacotes} \)]
    \glsentrydesc{indice:pacotes},
    tal que \(  \gls{indice:pacotes} \in [1, N] \);

    \item[\( \gls{indice:pedidos} \)]
    \glsentrydesc{indice:pedidos},
    tal que \(  \gls{indice:pedidos} \in [1, N] \);
\end{symbols}

\subsection{Constantes}

\subsubsection{Fixas}

As seguinte constante é fixa e não pode ser alterada durante a execução do modelo.

\begin{symbols}
    \item[\( \gls{constante:fixa:tempo-turno} \) ]
    \glsentrydesc{constante:fixa:tempo-turno},
    tal que \(  \gls{constante:fixa:tempo-turno} \in \mathbb{N}^{+} \).
\end{symbols}

O valor da constante citada e sua unidade são exibidos no \autoref{qua:constantes-fixas}.

\begin{quadro}
    \caption{%
        \label{qua:constantes-fixas}%
        Constantes do problema.
    }

    \begin{tabular}{|c|c|c|}
        \hline
        Constante                              &
        Valor                                  &
        Unidade
        \\
        \hline
        \( \gls{constante:fixa:tempo-turno} \) &
        420                                    &
        minutos
        \\
        \hline
    \end{tabular}

    \ComponenteFontePropria{}
\end{quadro}

\subsubsection{Dados}

\begin{symbols}
    \item[\( \gls{constante:dado:receita:rendimento} \)]
    \glsentrydesc{constante:dado:receita:rendimento},
    tal que \( \glsentryname{constante:dado:receita:rendimento} \in \mathbb{R}^{+} \);

    \item[\( \gls{constante:dado:receita:rendimento-total} \)]
    \glsentrydesc{constante:dado:receita:rendimento-total},
    tal que \( \glsentryname{constante:dado:receita:rendimento-total} \in \mathbb{R}^{+} \);

    \item[\( \gls{constante:dado:receita:tempo-mistura} \)]
    \glsentrydesc{constante:dado:receita:tempo-mistura},
    tal que \( \glsentryname{constante:dado:receita:tempo-mistura} \in \mathbb{N}^{+} \);

    \item[\( \gls{constante:dado:pacote:gramatura} \)]
    \glsentrydesc{constante:dado:pacote:gramatura},
    tal que \( \glsentryname{constante:dado:pacote:gramatura} \in \mathbb{N}^{+} \);

    \item[\( \gls{constante:dado:pedido:peso} \)]
    \glsentrydesc{constante:dado:pedido:peso},
    tal que \( \glsentryname{constante:dado:pedido:peso} \in \mathbb{R}^{+} \).
\end{symbols}

\subsection{Variáveis}

\subsubsection{Auxiliares}

Todas as variáveis auxiliares acrescentam uma coluna à matriz de dados, sendo calculadas a partir de outras variáveis.

As variáveis auxiliares calculadas a partir do problema são:

\begin{symbols}
    \item[\( \gls{variavel:auxiliar:peso-total-pedidos-atendidos} \)]
    \glsentrydesc{variavel:auxiliar:peso-total-pedidos-atendidos},
    tal que \( \glsentryname{variavel:auxiliar:peso-total-pedidos-atendidos} \in \mathbb{R} \);

    \item[\( \gls{variavel:auxiliar:peso-total-produzido} \)]
    \glsentrydesc{variavel:auxiliar:peso-total-produzido},
    tal que \( \glsentryname{variavel:auxiliar:peso-total-produzido} \in \mathbb{R} \);

    \item[\( \gls{variavel:auxiliar:peso-total-em-sobra} \)]
    \glsentrydesc{variavel:auxiliar:peso-total-em-sobra},
    tal que \( \glsentryname{variavel:auxiliar:peso-total-em-sobra} \in \mathbb{R} \);
\end{symbols}

\subsubsection{Principais}

\begin{symbols}
    \item[\( \gls{variavel:principal:quantidade-porcoes-receita-a-produzir} \)]
    \glsentrydesc{variavel:principal:quantidade-porcoes-receita-a-produzir},
    tal que \(  \glsentryname{variavel:principal:quantidade-porcoes-receita-a-produzir} \in \mathbb{N} \).
\end{symbols}

\section{Restrições}

\subsection{Auxiliares}

Todas as variáveis auxiliares incorrem em restrições de igualdade, uma vez que são calculadas segundo colunas da matriz de dados.

As restrições de igualdade que definem as variáveis auxiliares são:

\begin{align}
    \gls{variavel:auxiliar:peso-total-produzido} &
    = \gls{variavel:principal:quantidade-porcoes-receita-a-produzir} \times \gls{constante:dado:receita:rendimento}.
    \\
    \gls{variavel:auxiliar:peso-total-em-sobra}  &
    = \gls{variavel:auxiliar:peso-total-produzido} - \gls{variavel:auxiliar:peso-total-pedidos-atendidos}.
    \\
    \sum \gls{constante:dado:rendimento-total}   &
    \geq
    \sum \gls{constante:dado:pedido:peso}
    \\
    \gls{constante:dado:receita:tempo-mistura}   &
    \leq 420
\end{align}

Tempo máximo de mistura por expediente:

\begin{align}
    \gls{variavel:principal:quantidade-porcoes-receita-a-produzir}
    \times \gls{constante:dado:receita:tempo-mistura}
    \leq \gls{constante:fixa:tempo-turno}.
\end{align}

\section{Função-Objetivo}

A função objetivo deste problema é minimizar o peso total em sobra das receitas, conforme mostrado abaixo:

\begin{align}
    \min \sum \gls{variavel:auxiliar:peso-total-em-sobra}_{\gls{indice:pedidos}}
\end{align}


\section{Trabalhos Relacionados}%
\label{sec:relacionados}

\todo{Afazer}

Este trabalho foi realizado no \gls{dcc}, da unidade acadêmica \gls{ice}, da \gls{ufjf}.
Seu objetivo é tratar sobre \glspl{ga}, como introduzidos pela disciplina \gls{disciplina}.
Abordamos os temas de \gls{fitness}, \gls{crossover}, \gls{conjunto_vazio}, \glspl{fn}.


A \autoref{tab:classes_de_equivalencia_por_particionamento} apresenta as classes de equivalência obtidas por meio de particionamento das condições de entrada.

\begin{table}[htb]
    \IBGEtab{%
        \caption{Classes de equivalência obtidas por meio de particionamento das condições de entrada}%
        \label{tab:classes_de_equivalencia_por_particionamento}
    }{%
        \begin{tabular}{p{4.25cm}p{4.25cm}p{4.25cm}}
            \toprule
            \textbf{Condição de entrada}
             &
            \textbf{Classes de equivalência válidas}
             &
            \textbf{Classes de equivalência inválidas}
            \\

            \midrule

            \multirow{2}{4.25cm}{Estado atual da célula (\textcolor{Blue}{E})}
             &
            O estado atual da célula é vivo (\textcolor{Green}{V1})
             &
            \
            \\

            \
             &
            O estado atual da célula é morto (\textcolor{Green}{V2})
             &
            \
            \\

            \midrule

            \multirow{2}{4.25cm}{Quantidade de vizinhos vivos (\textcolor{Blue}{V})}
             &
            \multirow{2}{4.25cm}{Está no intervalo \( 0 \leq \mathcolor{Blue}{V} \leq 8 \) (\textcolor{Green}{V3})}
             &
            Abaixo do limite inferior: \( \mathcolor{Blue}{V} < 0 \) (\textcolor{Red}{I1})
            \\

            \
             &
            \
             &
            Acima do limite superior: \( \mathcolor{Blue}{V} > 8 \) (\textcolor{Red}{I2})
            \\

            \bottomrule
        \end{tabular}%
    }{%
        \fonte{\ComponenteFontePropria}%
    }
\end{table}

\subsection{Teste das variáveis}%
\label{sec:teste_variaveis}

\testaVariaveis{}


\chapter{Resultados}%
\label{cap:resultados}%
\index{Índice para os resultados}

\begin{figure}%
    \caption{\label{fig:f1}Quadrado preto}%
    \centering
    \includegraphics[scale=0.5]{imagens/black-square.png}
    \legend{Fonte:~\citeonline{tortinhas:2024}}
\end{figure}

\lipsum[1-40] % chktex-file 8


\section{Conclusão}%
\label{cap:conclusao}

% =====

% =====
% ELEMENTOS PÓS-TEXTUAIS
% =====
\postextual{}

% --- Referências ---
\bibliography{cx_bibliografia}

% --- Glossário ---
\printglossary[type=main,style=altlist,title=Glossário]
\printglossary[type=acronym,style=altlist,title=Lista de Abreviaturas e Siglas]
% =====

\end{document}