\documentclass[
    % --- Opções da classe memoir ---
	openright,			% Define que capítulos começam em página ímpar (insere página vazia caso preciso).
    12pt,				% Tamanho da fonte.
    %oneside,			% Para impressão apenas no anverso. Oposto a twoside.
    twoside,             % Para impressão em verso e anverso. Caso necessário, páginas do verso ficarão em branco. Oposto a oneside.
    a4paper,			% tamanho do papel.
    % -----
    % --- Opções da classe abntex2 ---
    %chapter=TITLE,		% Títulos de capítulos são convertidos para letras maiúsculas.
    %section=TITLE,		% Títulos de seções são convertidos para letras maiúsculas.
    %subsection=TITLE,	% Títulos de subseções são convertidos para letras maiúsculas.
    %subsubsection=TITLE, % Títulos de subsubseções são convertidos para letras maiúsculas.
    % -----
    % --- Opções do pacote babel ---
    english,			% Define idioma adicional para hifenização
    french,			    % Define idioma adicional para hifenização
    spanish,			% Define idioma adicional para hifenização
    brazil,				% O último idioma é definido como o principal do documento.
]{abntex2}

% --- Pacotes ---
% -----
% Pacotes fundamentais 
% -----

\usepackage{lmodern}        % Usa a fonte Latin Modern.
\usepackage[T1]{fontenc}    % Seleção de códigos de fonte.
\usepackage[utf8]{inputenc} % Codificação do documento (conversão automática dos acentos).
\usepackage{indentfirst}    % Identa o primeiro parágrafo de cada seção.
\usepackage{microtype}      % Para melhorias de justificação.

\usepackage{color}      % Controle das cores.
\usepackage{graphicx}   % Inclusão de gráficos.

\usepackage[brazilian,hyperpageref]{backref}    % Páginas com as citações na bibliografia.

\usepackage[
    abbreviations,
    symbols,
    nohypertypes={index},
    % nonumberlist=true,
    seeautonumberlist,
    subentrycounter,
    toc=false
]{glossaries-extra}   % Glossário. Permite definir termos, siglas e abreviações. Deve ser carregado depois de backref.

\usepackage[alf]{abntex2cite}       % Citações padrão ABNT. Deve ser carregado depois de glossaries.

\usepackage[portuguese]{todonotes}  % Adiciona notas e afazeres no documento.

\usepackage{multirow}   % Permite mesclar células em tabelas.
\usepackage{longtable}  % Permite criar tabelas com mais de uma página.

\usepackage{lipsum}     % Gera texto de preenchimento.

\usepackage{titlecaps}  % Fornece ferramentas para capitalização de palavras.

\usepackage{multicol}   % Permite incluir ambientes que exibem texto em múltiplas colunas.

% -----

% -----
% Configurações dos pacotes
% -----

% --- Glossaries-extra ---
% O pacote "glossaries-extra" provê a opção de criar um glossário chamado "index" nas opções do pacote. Entretanto, isso cria um comando chamado "\printindex", que conflita com o homônimo do pacote "memoir", utilizado para imprimir o índice remissivo. Para resolver, não habilitamos a opção "index" e criamos um novo glossário chamado "index" manualmente.
\newglossary[ilg]{index}{ind}{idx}{\indexname}
% Define o comando "\newterm" para facilitar a criação de novos termos no glossário "index".
\DeclareDocumentCommand\newterm{m m g}{%
    \IfValueTF{#3}{%
        \newglossaryentry{#1}{%
            type=index,
            name={#2},
            description={\nopostdesc},
            plural={#3}%
        }
    }{%
        \newglossaryentry{#1}{%
            type=index,
            name={#2},
            description={\nopostdesc}%
        }
    }
}
% Permite a impressão dos glossários.
\makeglossaries{}
\loadglsentries{c0_glossario.tex}
\renewcommand*{\glsclearpage}{} % Evita quebra de página entre os glossários.

% --- Backref ---
% Usado sem a opção hyperpageref de backref.
\renewcommand{\backrefpagesname}{Citado na(s) página(s):~}
% Texto padrão antes do número das páginas.
\renewcommand{\backref}{}
% Define os textos da citação.
\renewcommand*{\backrefalt}[4]{%
    \ifcase #1%
        Nenhuma citação no texto.%
    \or%
        Citado na página #2.%
    \else%
        Citado #1 vezes nas páginas #2.%
    \fi}%

% --- Todonotes ---
% Define a largura da caixa de notas.
\setlength{\marginparwidth}{2cm}
% Define a cor e o estilo da caixa de notas.
\presetkeys{todonotes}{inline,backgroundcolor=yellow}{}
% Desabilita as notas.
% \presetkeys{todonotes}{disable}{}

% --- Titlecaps ---
% Define as palavras que não devem ser capitalizadas. Elas ainda serão capitalizadas se estiverem no início de uma frase.
\Addlcwords{a o e um uma uns umas em de da do das dos no na nos nas num numa nuns numas por para com pelo pela pelos pelas}

% -----


% --- Comandos ---
% --- Variáveis ---
\input{definicoes/cd_variaveis.tex}

% --- Componentes ---
\input{definicoes/cd_componentes.tex}


% --- Hifenização ---
\input{c0_hifenizacao.tex}

% --- Definições pre-textuais ---
% -----
% Informações de dados para gerar os ARTEFATOS
% -----

% --- Título ---
% Título principal do trabalho. Parâmetros: título.
\titulo{Título do meu trabalho}
% Subtítulo do trabalho. Parâmetros: subtítulo (opcional).
\subtitulo{O subtítulo do meu trabalho}

% --- Trabalho ---
% Tipo de trabalho. Parâmetros: classificação (doutorado, mestrado, especializacao, licenciatura, bacharelado).
\tipo{bacharelado}

% Curso. Parâmetros: nome do curso.
\curso{Sistemas de Informação}

% -- Data da aprovação --
% Dia. Parâmetros: dia no formato 01.
\dia{01}
% Mês. Parâmetros: mês por extenso, em letras minúsculas.
\mes{janeiro}
% Ano. Parâmetros: ano no formato 1000.
\ano{2025}

% -- Local de publicação --
% Cidade. Parâmetros: nome da cidade.
\cidade{Juiz de Fora}
% Estado. Parâmetros: nome do estado.
\estado{Minas Gerais}
% País. Parâmetros: nome do país.
\pais{Brasil}

% --- Instituição ---
% Instituição. Parâmetros: nome da instituição; sigla da instituição; cidade da instituição.
\instituicao{Universidade Federal de Juiz de Fora}{UFJF}{Juiz de Fora}
% Unidade acadêmica. Parâmetros: nome da unidade acadêmica; sigla da unidade acadêmica.
\unidadeAcademica{Instituto de Ciências Exatas}{ICE}
% Departamento. Parâmetros: nome do departamento; sigla do departamento.
\departamento{Departamento de Ciência da Computação}{DCC}

% --- Pessoas ---
% Autor. Parâmetros: Último sobrenome; restante do nome.
\autor{Alves}{Alice Carvalho de}

% Orientador. Parâmetros: título; último sobrenome; restante do nome; instituição de afiliação.
\orientador{Profa.\ Dra.}{Batista}{Beatriz Alves}{Universidade Federal de Juiz de Fora}

% Coorientador. Parâmetros: título; último sobrenome; restante do nome; instituição de afiliação.
\coorientador{Prof.\ Dr.}{Carvalho}{Carlos Ribeiro de}{Universidade Federal de Juiz de Fora}

% Examinador um. Parâmetros: título; último sobrenome; restante do nome; instituição de afiliação.
\examinadorUm{Prof.\ Dr.}{Dias}{Daniel Pereira}{Universidade Federal de Juiz de Fora}

% Examinador dois. Parâmetros: título; último sobrenome; restante do nome; instituição de afiliação.
\examinadorDois{Profa.\ Dra.}{Espinoza}{Eva Maria}{Universidade Federal de Juiz de Fora}

% Examinador três. Parâmetros: título; último sobrenome; restante do nome; instituição de afiliação.
\examinadorTres{Prof.\ Dr.}{Ferreira}{Fernando dos Santos}{Universidade Federal de Juiz de Fora}

% Examinador quatro. Parâmetros: título; último sobrenome; restante do nome; instituição de afiliação.
\examinadorQuatro{Profa.\ Dra.}{Gonçalves}{Gabriela Almeida}{Universidade Federal de Juiz de Fora}

% --- Resumo ---
% Palavras chave. Parâmetros: palavra-chave um; palavra-chave dois; palavra-chave três; palavra-chave quatro (opcional); palavra-chave cinco (opcional). Os parâmetros opcionais devem ser envolvidos por colchetes. Todas as palavras devem estar em letras minúsculas, salvo para siglas.
\palavrasChave{palavra um}{palavra dois}{palavra três}[palavra quatro]

% -----


% --- Configurações da aparência do documento ---
\input{definicoes/cd_aparencia.tex}

% Inicia o documento.
\begin{document}

% -----
% Configurações do texto
% -----
% Seleciona o idioma do documento.
\selectlanguage{brazil}

% Define a mesma largura para os espaços entre palavras de uma mesma sentença e entre sentenças diferentes. 
\frenchspacing{}
% -----

% =====
% ELEMENTOS PRÉ-TEXTUAIS
% =====
\pretextual{}

% --- Capa ---
% -----
% Capa
% -----

\renewcommand{\imprimircapa}{%
    \begin{capa}%
        \center%
        \ABNTEXchapterfont%

        {%
            \bfseries
            \MakeUppercase{\ValorDoNomeDaInstituicao{}}
            \par
            \MakeUppercase{\ValorDoNomeDaUnidadeAcademica{}}
            \par
            \MakeUppercase{\ValorDoPrograma{}}
            \par
        }
        \vspace{4cm}

        \ComponenteAutor{}
        \vspace{4cm}

        \ComponenteTituloESubtitulo{}
        \vspace{\fill}

        \ComponenteLocalEData{}
    \end{capa}
}

\imprimircapa{}

% -----


% --- Folha de rosto ---
% -----
% Folha de rosto
% -----

\makeatletter
\renewcommand{\folhaderostocontent}{%
    \ABNTEXchapterfont%
    \center%

    \ComponenteAutor{}
    \vspace{4cm}

    \ComponenteTituloESubtitulo{}
    \vspace{4cm}

    \ComponenteNatureza{}
    \vspace{2cm}

    \begin{flushright}
        \large
        Orientador: \ValorDoNomeCompletoDoOrientadorComTitulo{}
        \par
        \ifDefinidoCoorientador%
            Coorientador: \ValorDoNomeCompletoDoCoorientadorComTitulo{}
            \par
        \fi
    \end{flushright}
    \vfill

    \ComponenteLocalEData{}
}
\makeatother

% -----

% Imprime a folha de rosto com a ficha catalográfica no verso.
\imprimirfolhaderosto*
% Imprime a folha de rosto com a ficha catalográfica em outra página.
% \imprimirfolhaderosto{}

% --- Ficha catalográfica ---
\input{artefatos/ca_ficha.tex}

% --- Errata ---
\input{artefatos/ca_errata.tex}

% --- Folha de aprovação ---
% -----
% Folha de aprovação
% -----

% Para a versão final do trabalho, sua instituição pode lhe fornecer um PDF com a versão definitiva da ficha catalográfica.
% Se for este o caso, salve o PDF no diretório do seu projeto e substitua todo o conteúdo deste arquivo por:

% \begin{folhadeaprovacao}
% \includepdf{folhadeaprovacao_final.pdf}
% \end{folhadeaprovacao}

\begin{folhadeaprovacao}
    \ABNTEXchapterfont%
    \center%

    {%
        \large
        \MakeUppercase{\ValorDoNomeCompletoDoAutor{}}
    }
    \vspace*{\fill}

    {%
        \Large
        \textbf{\MakeUppercase{\ValorDoTitulo{}}}%
        \par
        \ifDefinidoSubtitulo%
            \ValorDoSubtitulo{}%
            \par
        \fi
    }
    \vspace*{\fill}

    \hspace{.45\textwidth}
    \begin{minipage}{.5\textwidth}
        \ValorDoTipo{}\ submetida ao corpo docente do \ValorDaUnidadeAcademica{}\ da \ValorDaInstituicao{}, como
        parte integrante dos requisitos necess\'arios para a obten\c{c}\~ao do grau de \ValorDoGrau{}\ em \ValorDoCurso{}.
    \end{minipage}%
    \vspace*{\fill}

    {%
        Trabalho aprovado. \ValorDaCidade{}, \ValorDoDia{} de \ValorDoMes{} de \ValorDoAno{}:
    }

    \begin{multicols}{2}
        \setlength{\ABNTEXsignwidth}{7cm}%

        \assinatura*{%
            \textbf\ValorDoNomeCompletoDoOrientadorComTitulo{}\\
            \ValorDaInstituicaoDoOrientador{}\\
            Orientador\\
        }

        \ifDefinidoCoorientador%
            \assinatura*{%
                \textbf\ValorDoNomeCompletoDoCoorientadorComTitulo{}\\
                \ValorDaInstituicaoDoCoorientador{}\\
                Coorientador
            }
        \fi

        \ifDefinidoExaminadorUm%
            \assinatura*{%
                \textbf\ValorDoNomeCompletoDoExaminadorUmComTitulo{}\\
                \ValorDaInstituicaoDoExaminadorUm{}\\
                Examinador
            }
        \fi

        \ifDefinidoExaminadorDois%
            \assinatura*{%
                \textbf\ValorDoNomeCompletoDoExaminadorDoisComTitulo{}\\
                \ValorDaInstituicaoDoExaminadorDois{}\\
                Examinador
            }
        \fi

        \ifDefinidoExaminadorTres%
            \assinatura*{%
                \textbf\ValorDoNomeCompletoDoExaminadorTresComTitulo{}\\
                \ValorDaInstituicaoDoExaminadorTres{}\\
                Examinador
            }
        \fi

        \ifDefinidoExaminadorQuatro%
            \assinatura*{%
                \textbf\ValorDoNomeCompletoDoExaminadorQuatroComTitulo{}\\
                \ValorDaInstituicaoDoExaminadorQuatro{}\\
                Examinador
            }
        \fi

    \end{multicols}

    {%
    \MakeUppercase{\ValorDaCidade{}}
    \par
    \MakeUppercase{\ValorDoMes{}},
    \ValorDoAno{}\\
    }
\end{folhadeaprovacao}

% -----


% --- Dedicatória ---
% -----
% Dedicatória
% -----

\begin{dedicatoria}
    \vspace*{\fill}
    \itshape
    \begin{flushright}

        Este trabalho é dedicado às crianças adultas que, quando pequenas, sonharam em se tornar cientistas.

    \end{flushright}
\end{dedicatoria}

% -----


% --- Agradecimentos ---
% -----
% Agradecimentos
% -----

\begin{agradecimentos}

    Agradecemos ao projeto \abnTeX\footnote{%
        Acesso em:~\url{http://www.abntex.net.br/}.%
    }, que disponibiliza o modelo \LaTeX\ que foi customizado para a elaboração de trabalhos acadêmicos conforme as normas da ABNT.

    Agradecemos também ao professor Dr.\ Jairo Souza, que desenvolveu o modelo \LaTeX\ para o \gls{dcc} da \gls{ufjf}, o qual baseou partes deste modelo.

    \lipsum[17]

\end{agradecimentos}

% -----


% --- Epígrafe ---
% -----
% Epígrafe
% -----

\begin{epigrafe}
    \vspace*{\fill}
    \itshape
    \begin{flushright}

        ``Mathematical reasoning may be regarded rather schematically as the exercise of a combination of two facilities, which we may call intuition and ingenuity'' (Alan Turing, Systems of Logic Based on Ordinals).

    \end{flushright}
\end{epigrafe}

% -----


% --- Resumos ---
% --- Português ---
\begin{resumoumacoluna}

    % Escreva aqui o resumo em português.
    \lipsum[1-3]

    \vspace{\onelineskip}
    \noindent
    \textbf{Palavras-chave}:
    % As palavras-chave em português devem ser definidas no arquivo de pretexto.
    \ValorDasPalavrasChave{}.
\end{resumoumacoluna}

% --- Língua estrangeira ---
\renewcommand{\resumoname}{Abstract}
\begin{resumoumacoluna}
    \begin{otherlanguage*}{english}

        % Escreva aqui o resumo em língua estrangeira.
        \lipsum[4-6]

        \vspace{\onelineskip}
        \noindent
        \textbf{Keywords}:
        % Escreva aqui as palavras-chave em língua estrangeira.
        word one, word two, word three, word four, word five.
    \end{otherlanguage*}
\end{resumoumacoluna}

% \begin{resumoumacoluna}[Résumé]
%     \begin{otherlanguage*}{french}
%         Il s'agit d'un résumé en français.

%         \vspace{\onelineskip}
%         \noindent
%         \textbf{Mots-clés}:
%         latex. abntex. publication de textes.
%     \end{otherlanguage*}
% \end{resumoumacoluna}

% \begin{resumoumacoluna}[Resumen]
%     \begin{otherlanguage*}{spanish}
%         Este es el resumen en español.

%         \vspace{\onelineskip}
%         \noindent
%         \textbf{Palabras clave}: latex. abntex. publicación de textos.
%     \end{otherlanguage*}
% \end{resumoumacoluna}


% --- Lista de ilustrações ---
\pdfbookmark[0]{\listfigurename}{lof}
\listoffigures*
\cleardoublepage{}

% --- Lista de tabelas ---
\pdfbookmark[0]{\listtablename}{lot}
\listoftables*
\cleardoublepage{}

% --- Lista de abreviaturas e siglas ---
\printglossary[type=abbreviations,style=altlist,title=\listadesiglasname]
\cleardoublepage{}

% --- Lista de símbolos ---
\printglossary[type=symbols,title=\listadesimbolosname]
\cleardoublepage{}

% --- Sumário ---
\pdfbookmark[0]{\contentsname}{toc}
\tableofcontents*
\cleardoublepage{}

% =====

% =====
% ELEMENTOS TEXTUAIS
% =====
\textual{}

\chapter{Introdução}%
\label{cap:introducao}

Este modelo se baseia no Modelo Canônico para trabalhos Acadêmicos do projeto \abnTeX~\cite{abntex2:2024}.

Também se utilizam recursos desenvolvidos pelo projeto de \citeonline{souza:2024}, que adapta as regras de formatação da \gls{ufjf} para um modelo em \LaTeX.

Por fim, as definições foram ajustadas conforme o Manual de normalização e modelos de trabalhos acadêmicos da \gls{ufjf}~\cite{cdd:2023}.


\chapter{Fundamentação Teórica}%
\label{cap:fundamentacao}

Apresentamos exemplos de uso de símbolos matemáticos e equações.

\section{Dados}

\subsection{Índices}

O problema permite delimitar os seguintes índices:

\begin{symbols}
    \item[\( \gls{indice:receitas} \)]
    \glsentrydesc{indice:receitas},
    tal que \(  \gls{indice:receitas} \in [1, N] \);

    \item[\( \gls{indice:pacotes} \)]
    \glsentrydesc{indice:pacotes},
    tal que \(  \gls{indice:pacotes} \in [1, N] \);

    \item[\( \gls{indice:pedidos} \)]
    \glsentrydesc{indice:pedidos},
    tal que \(  \gls{indice:pedidos} \in [1, N] \);
\end{symbols}

\subsection{Constantes}

\subsubsection{Fixas}

As seguinte constante é fixa e não pode ser alterada durante a execução do modelo.

\begin{symbols}
    \item[\( \gls{constante:fixa:tempo-turno} \) ]
    \glsentrydesc{constante:fixa:tempo-turno},
    tal que \(  \gls{constante:fixa:tempo-turno} \in \mathbb{N}^{+} \).
\end{symbols}

O valor da constante citada e sua unidade são exibidos no \autoref{qua:constantes-fixas}.

\begin{quadro}
    \caption{%
        \label{qua:constantes-fixas}%
        Constantes do problema.
    }

    \begin{tabular}{|c|c|c|}
        \hline
        Constante                              &
        Valor                                  &
        Unidade
        \\
        \hline
        \( \gls{constante:fixa:tempo-turno} \) &
        420                                    &
        minutos
        \\
        \hline
    \end{tabular}

    \ComponenteFontePropria{}
\end{quadro}

\subsubsection{Dados}

\begin{symbols}
    \item[\( \gls{constante:dado:receita:rendimento} \)]
    \glsentrydesc{constante:dado:receita:rendimento},
    tal que \( \glsentryname{constante:dado:receita:rendimento} \in \mathbb{R}^{+} \);

    \item[\( \gls{constante:dado:receita:rendimento-total} \)]
    \glsentrydesc{constante:dado:receita:rendimento-total},
    tal que \( \glsentryname{constante:dado:receita:rendimento-total} \in \mathbb{R}^{+} \);

    \item[\( \gls{constante:dado:receita:tempo-mistura} \)]
    \glsentrydesc{constante:dado:receita:tempo-mistura},
    tal que \( \glsentryname{constante:dado:receita:tempo-mistura} \in \mathbb{N}^{+} \);

    \item[\( \gls{constante:dado:pacote:gramatura} \)]
    \glsentrydesc{constante:dado:pacote:gramatura},
    tal que \( \glsentryname{constante:dado:pacote:gramatura} \in \mathbb{N}^{+} \);

    \item[\( \gls{constante:dado:pedido:peso} \)]
    \glsentrydesc{constante:dado:pedido:peso},
    tal que \( \glsentryname{constante:dado:pedido:peso} \in \mathbb{R}^{+} \).
\end{symbols}

\subsection{Variáveis}

\subsubsection{Auxiliares}

Todas as variáveis auxiliares acrescentam uma coluna à matriz de dados, sendo calculadas a partir de outras variáveis.

As variáveis auxiliares calculadas a partir do problema são:

\begin{symbols}
    \item[\( \gls{variavel:auxiliar:peso-total-pedidos-atendidos} \)]
    \glsentrydesc{variavel:auxiliar:peso-total-pedidos-atendidos},
    tal que \( \glsentryname{variavel:auxiliar:peso-total-pedidos-atendidos} \in \mathbb{R} \);

    \item[\( \gls{variavel:auxiliar:peso-total-produzido} \)]
    \glsentrydesc{variavel:auxiliar:peso-total-produzido},
    tal que \( \glsentryname{variavel:auxiliar:peso-total-produzido} \in \mathbb{R} \);

    \item[\( \gls{variavel:auxiliar:peso-total-em-sobra} \)]
    \glsentrydesc{variavel:auxiliar:peso-total-em-sobra},
    tal que \( \glsentryname{variavel:auxiliar:peso-total-em-sobra} \in \mathbb{R} \);
\end{symbols}

\subsubsection{Principais}

\begin{symbols}
    \item[\( \gls{variavel:principal:quantidade-porcoes-receita-a-produzir} \)]
    \glsentrydesc{variavel:principal:quantidade-porcoes-receita-a-produzir},
    tal que \(  \glsentryname{variavel:principal:quantidade-porcoes-receita-a-produzir} \in \mathbb{N} \).
\end{symbols}

\section{Restrições}

\subsection{Auxiliares}

Todas as variáveis auxiliares incorrem em restrições de igualdade, uma vez que são calculadas segundo colunas da matriz de dados.

As restrições de igualdade que definem as variáveis auxiliares são:

\begin{align}
    \gls{variavel:auxiliar:peso-total-produzido} &
    = \gls{variavel:principal:quantidade-porcoes-receita-a-produzir} \times \gls{constante:dado:receita:rendimento}.
    \\
    \gls{variavel:auxiliar:peso-total-em-sobra}  &
    = \gls{variavel:auxiliar:peso-total-produzido} - \gls{variavel:auxiliar:peso-total-pedidos-atendidos}.
    \\
    \sum \gls{constante:dado:rendimento-total}   &
    \geq
    \sum \gls{constante:dado:pedido:peso}
    \\
    \gls{constante:dado:receita:tempo-mistura}   &
    \leq 420
\end{align}

Tempo máximo de mistura por expediente:

\begin{align}
    \gls{variavel:principal:quantidade-porcoes-receita-a-produzir}
    \times \gls{constante:dado:receita:tempo-mistura}
    \leq \gls{constante:fixa:tempo-turno}.
\end{align}

\section{Função-Objetivo}

A função objetivo deste problema é minimizar o peso total em sobra das receitas, conforme mostrado abaixo:

\begin{align}
    \min \sum \gls{variavel:auxiliar:peso-total-em-sobra}_{\gls{indice:pedidos}}
\end{align}


\section{Trabalhos Relacionados}%
\label{sec:relacionados}

\todo{Afazer}

Este trabalho foi realizado no \gls{dcc}, da unidade acadêmica \gls{ice}, da \gls{ufjf}.
Seu objetivo é tratar sobre \glspl{ga}, como introduzidos pela disciplina \gls{disciplina}.
Abordamos os temas de \gls{fitness}, \gls{crossover}, \gls{conjunto_vazio}, \glspl{fn}.


A \autoref{tab:classes_de_equivalencia_por_particionamento} apresenta as classes de equivalência obtidas por meio de particionamento das condições de entrada.

\begin{table}[htb]
    \IBGEtab{%
        \caption{Classes de equivalência obtidas por meio de particionamento das condições de entrada}%
        \label{tab:classes_de_equivalencia_por_particionamento}
    }{%
        \begin{tabular}{p{4.25cm}p{4.25cm}p{4.25cm}}
            \toprule
            \textbf{Condição de entrada}
             &
            \textbf{Classes de equivalência válidas}
             &
            \textbf{Classes de equivalência inválidas}
            \\

            \midrule

            \multirow{2}{4.25cm}{Estado atual da célula (\textcolor{Blue}{E})}
             &
            O estado atual da célula é vivo (\textcolor{Green}{V1})
             &
            \
            \\

            \
             &
            O estado atual da célula é morto (\textcolor{Green}{V2})
             &
            \
            \\

            \midrule

            \multirow{2}{4.25cm}{Quantidade de vizinhos vivos (\textcolor{Blue}{V})}
             &
            \multirow{2}{4.25cm}{Está no intervalo \( 0 \leq \mathcolor{Blue}{V} \leq 8 \) (\textcolor{Green}{V3})}
             &
            Abaixo do limite inferior: \( \mathcolor{Blue}{V} < 0 \) (\textcolor{Red}{I1})
            \\

            \
             &
            \
             &
            Acima do limite superior: \( \mathcolor{Blue}{V} > 8 \) (\textcolor{Red}{I2})
            \\

            \bottomrule
        \end{tabular}%
    }{%
        \fonte{\ComponenteFontePropria}%
    }
\end{table}

\subsection{Teste das variáveis}%
\label{sec:teste_variaveis}

\testaVariaveis{}


\chapter{Resultados}%
\label{cap:resultados}%
\index{Índice para os resultados}

\begin{figure}%
    \caption{\label{fig:f1}Quadrado preto}%
    \centering
    \includegraphics[scale=0.5]{imagens/black-square.png}
    \legend{Fonte:~\citeonline{tortinhas:2024}}
\end{figure}

\lipsum[1-40] % chktex-file 8


\section{Conclusão}%
\label{cap:conclusao}

% =====

% =====
% ELEMENTOS PÓS-TEXTUAIS
% =====
\postextual{}

% --- Referências ---
\bibliography{cx_bibliografia}

% --- Glossário ---
\printglossary[type=main,style=altlist,title=Glossário]
\addcontentsline{toc}{chapter}{\glossaryname}

% --- Apêndices ---
% -----
% Apêndices
% -----

\begin{apendicesenv}
    % Imprime uma página indicando o início dos apêndices.
    \partapendices%

    \input{apendices/c1_quisque.tex}
    \input{apendices/c2_nullam.tex}

\end{apendicesenv}

% -----


% --- Anexos ---
% -----
% Anexos
% -----

\begin{anexosenv}
    % Imprime uma página indicando o início dos anexos.
    \partanexos%

    \chapter{Morbi ultrices rutrum lorem.}%
\label{anx:morbi}

\lipsum[30]

    \input{anexos/c2_cras.tex}
    \chapter{Fusce facilisis lacinia dui}%
\label{anx:fusce}

\lipsum[32]


\end{anexosenv}

% -----


% --- Índice remissivo ---
\phantompart{}
\printindex{}
% =====

\end{document}
