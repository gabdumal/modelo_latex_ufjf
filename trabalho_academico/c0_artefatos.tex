% =====
% Conteúdo dos artefatos
% =====

% -----
% Folha de rosto
% -----


% -----

% -----
% Ficha catalográfica
% -----

% Para a versão final do trabalho, sua instituição pode lhe fornecer um PDF com a versão definitiva da ficha catalográfica.

% Se for este o caso, salve o PDF no diretório "documentos" do seu projeto, então utilize a segunda versão do comando abaixo.

% Substitua "documentos/black-square.pdf" pelo caminho do PDF que você salvou no seu projeto.

\inserirFichaCatalografica%
% \inserirFichaCatalografica[%
%     \includepdf{documentos/black-square.pdf}
% ]

% -----

% -----
% Resumo
% -----

% --- Língua vernácula ---

% Palavras chave. Parâmetros: palavra-chave um; palavra-chave dois; palavra-chave três; palavra-chave quatro (opcional); palavra-chave cinco (opcional). Deixe em branco os campos opcionais que não deseja preencher.
\palavrasChave{palavra um}{palavra dois}{palavra três}{palavra quatro}{}

\conteudoDoResumo{%
    \lipsum[1-3]
}

% --- Língua estrangeira ---

% Parâmetros: língua (english, french, spanish, italian, german, dutch); palavras-chave; conteúdo.

\insereResumoEmLinguaEstrangeira%
{english}
{\formataPalavrasChave{word one}{word two}{word three}{word four}{}}
{%
    \lipsum[4-6]
}

\insereResumoEmLinguaEstrangeira%
{french}
{\formataPalavrasChave{word one}{word two}{word three}{word four}{}}
{%
    \lipsum[30-32]
}

% -----

% =====
