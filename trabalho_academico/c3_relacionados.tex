\chapter{Trabalhos Relacionados}%
\label{cap:relacionados}

\todo{Afazer}

Este trabalho foi realizado no \gls{dcc}, da unidade acadêmica \gls{ice}, da \gls{ufjf}.
Seu objetivo é tratar sobre \glspl{ga}, como introduzidos pela disciplina \gls{disciplina}.
Abordamos os temas de \gls{fitness}, \gls{crossover}, \gls{conjunto_vazio}, \glspl{fn}.

A \autoref{tab:classes_de_equivalencia_por_particionamento} apresenta as classes de equivalência obtidas por meio de particionamento das condições de entrada.

\begin{table}[htb]
    \IBGEtab{%
        \caption{Classes de equivalência obtidas por meio de particionamento das condições de entrada}%
        \label{tab:classes_de_equivalencia_por_particionamento}
    }{%
        \begin{tabular}{p{4.25cm}p{4.25cm}p{4.25cm}}
            \toprule
            \textbf{Condição de entrada}
             &
            \textbf{Classes de equivalência válidas}
             &
            \textbf{Classes de equivalência inválidas}
            \\

            \midrule

            \multirow{2}{4.25cm}{Estado atual da célula (\textcolor{Blue}{E})}
             &
            O estado atual da célula é vivo (\textcolor{Green}{V1})
             &
            \
            \\

            \
             &
            O estado atual da célula é morto (\textcolor{Green}{V2})
             &
            \
            \\

            \midrule

            \multirow{2}{4.25cm}{Quantidade de vizinhos vivos (\textcolor{Blue}{V})}
             &
            \multirow{2}{4.25cm}{Está no intervalo \( 0 \leq \mathcolor{Blue}{V} \leq 8 \) (\textcolor{Green}{V3})}
             &
            Abaixo do limite inferior: \( \mathcolor{Blue}{V} < 0 \) (\textcolor{Red}{I1})
            \\

            \
             &
            \
             &
            Acima do limite superior: \( \mathcolor{Blue}{V} > 8 \) (\textcolor{Red}{I2})
            \\

            \bottomrule
        \end{tabular}%
    }{%
        \fonte{\ComponenteFontePropria}%
        \nota{Este é um exemplo de nota.}%
        \nota[Anotações]{Uma anotação adicional, seguida de várias outras.}%
    }
\end{table}

\section{Teste das variáveis}%
\label{sec:teste_variaveis}

\testaVariaveis{}

\lipsum[1-20] % chktex-file 8
