% -----
% Informações de dados para gerar os ARTEFATOS
% -----

% --- Título ---
% Título principal do trabalho. Parâmetros: título.
\titulo{Título do meu trabalho}
% Subtítulo do trabalho. Parâmetros: subtítulo (opcional).
\subtitulo{O subtítulo do meu trabalho}

% --- Trabalho ---
% Tipo de trabalho. Parâmetros: classificação (doutorado, mestrado, especializacao, licenciatura, bacharelado).
\tipo{bacharelado}

% Curso. Parâmetros: nome do curso.
\curso{Sistemas de Informação}

% -- Data da aprovação --
% Dia. Parâmetros: dia no formato 01.
\dia{01}
% Mês. Parâmetros: mês por extenso, em letras minúsculas.
\mes{janeiro}
% Ano. Parâmetros: ano no formato 1000.
\ano{2025}

% -- Local de publicação --
% Cidade. Parâmetros: nome da cidade.
\cidade{Juiz de Fora}
% Estado (Opcional). Parâmetros: nome do estado; sigla do estado.
% \estado{Minas Gerais}{MG}
% País. Parâmetros: nome do país.
\pais{Brasil}

% --- Instituição ---
% Instituição. Parâmetros: artigo gramatical do gênero (a,o); nome; sigla; cidade.
\instituicao{a}{Universidade Federal de Juiz de Fora}{UFJF}{Juiz de Fora}
% Unidade acadêmica. Parâmetros: artigo gramatical do gênero (a,o); nome; sigla.
\unidadeAcademica{o}{Instituto de Ciências Exatas}{ICE}
% Departamento. Parâmetros: artigo gramatical do gênero (a,o); nome; sigla.
\departamento{o}{Departamento de Ciência da Computação}{DCC}

% --- Pessoas ---
% Autor. Parâmetros: Último sobrenome; restante do nome.
\autor{Alves}{Alice Carvalho de}

% Orientador. Parâmetros: título; último sobrenome; restante do nome; instituição de afiliação.
\orientador{Profa.\ Dra.}{Batista}{Beatriz Alves}{Universidade Federal de Juiz de Fora}

% Coorientador. Parâmetros: título; último sobrenome; restante do nome; instituição de afiliação.
\coorientador{Prof.\ Dr.}{Carvalho}{Carlos Ribeiro de}{Universidade Federal de Juiz de Fora}

% Examinador um. Parâmetros: título; último sobrenome; restante do nome; instituição de afiliação.
\examinadorUm{Prof.\ Dr.}{Dias}{Daniel Pereira}{Universidade Federal de Juiz de Fora}

% Examinador dois. Parâmetros: título; último sobrenome; restante do nome; instituição de afiliação.
\examinadorDois{Profa.\ Dra.}{Espinoza}{Eva Maria}{Universidade Federal de Juiz de Fora}

% Examinador três. Parâmetros: título; último sobrenome; restante do nome; instituição de afiliação.
% \examinadorTres{Prof.\ Dr.}{Ferreira}{Fernando dos Santos}{Universidade Federal de Juiz de Fora}

% Examinador quatro. Parâmetros: título; último sobrenome; restante do nome; instituição de afiliação.
% \examinadorQuatro{Profa.\ Dra.}{Gonçalves}{Gabriela Almeida}{Universidade Federal de Juiz de Fora}

% --- Resumo ---
% Palavras chave. Parâmetros: palavra-chave um; palavra-chave dois; palavra-chave três; palavra-chave quatro (opcional); palavra-chave cinco (opcional). Os parâmetros opcionais devem ser envolvidos por colchetes. Todas as palavras devem estar em letras minúsculas, salvo para siglas.
\palavrasChave{palavra um}{palavra dois}{palavra três}[palavra quatro]

% -----
