% =====
% Resumo
% =====

% -----
% Português
% -----

\NewDocumentEnvironment{AmbienteResumo}{}
{%
    \resumo%
}
{%
    \par
    \vspace{\onelineskip}
    \noindent
    \textbf{Palavras-chave}:
    % As palavras-chave em português devem ser definidas no arquivo de pretexto. Elas serão separadas pelo caractere ponto e vírgula.
    \ValorDasPalavrasChave{}.
    \endresumo%
}

\def\conteudoDoResumo#1{
    \gdef\ValorDoConteudoDoResumo{#1}
}

% -----
% Língua estrangeira
% -----

\def\ValorDosResumosEmLinguasEstrangeiras{}

\NewDocumentCommand{\obtemTermoPalavrasChaveTraduzido}{m}{%
    \IfStrEqCase{#1}{%
        {english}{Keywords}%
            {french}{Mots-clés}%
            {spanish}{Palabras clave}%
            {italian}{Parole chiave}%
            {german}{Schlüsselwörter}%
            {dutch}{Trefwoorden}%
    }[Palavras-chave]%
}

\NewDocumentCommand{\obtemTermoResumoTraduzido}{m}{%
    \IfStrEqCase{#1}{%
        {english}{Abstract}%
            {french}{Résumé}%
            {spanish}{Resumen}%
            {italian}{Riassunto}%
            {german}{Zusammenfassung}%
            {dutch}{Samenvatting}%
    }[Resumo]%
}

% Parâmetros: língua; palavras-chave.
\NewDocumentEnvironment{AmbienteResumoEmLinguaEstrangeira}{mm}
{%
    \otherlanguage{#1}
    \resumo[\obtemTermoResumoTraduzido{#1}]
}
{%
    \par
    \vspace{\onelineskip}
    \noindent
    \textbf{\obtemTermoPalavrasChaveTraduzido{#1}}:
    % As palavras-chave serão separadas pelo caractere ponto e vírgula.
    #2.
    \endresumo%
    \endotherlanguage%
}

\NewDocumentCommand{\insereResumoEmLinguaEstrangeira}{mmm}{%
    \expandafter\def\expandafter\ValorDosResumosEmLinguasEstrangeiras\expandafter{\ValorDosResumosEmLinguasEstrangeiras% chkTeX 41
        \begin{AmbienteResumoEmLinguaEstrangeira}{#1}{#2}
            #3
        \end{AmbienteResumoEmLinguaEstrangeira}%
    }
}

% -----

% =====
