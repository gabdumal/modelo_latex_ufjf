% -----
% Pacotes fundamentais 
% -----

\usepackage{lmodern}        % Usa a fonte Latin Modern.
\usepackage[T1]{fontenc}    % Seleção de códigos de fonte.
\usepackage[utf8]{inputenc} % Codificação do documento (conversão automática dos acentos).
\usepackage{indentfirst}    % Identa o primeiro parágrafo de cada seção.
\usepackage{microtype}      % Para melhorias de justificação.

\usepackage{color}      % Controle das cores.
\usepackage{graphicx}   % Inclusão de gráficos.
\usepackage{amssymb}    % Símbolos matemáticos.
\usepackage{amsmath}    % Equações matemáticas.
\usepackage{amsfonts}   % Fontes e símbolos matemáticos.

\usepackage[brazilian,hyperpageref]{backref}    % Páginas com as citações na bibliografia.

\usepackage[
    abbreviations,
    symbols,
    nohypertypes={index},
    % nonumberlist=true,
    seeautonumberlist,
    subentrycounter,
    toc=false
]{glossaries-extra}   % Glossário. Permite definir termos, siglas e abreviações. Deve ser carregado depois de backref.

\usepackage[alf]{abntex2cite}       % Citações padrão ABNT. Deve ser carregado depois de glossaries.

\usepackage[portuguese]{todonotes}  % Adiciona notas e afazeres no documento.

\usepackage{multirow}   % Permite mesclar células em tabelas.
\usepackage{longtable}  % Permite criar tabelas com mais de uma página.

\usepackage{lipsum}     % Gera texto de preenchimento.

\usepackage{titlecaps}  % Fornece ferramentas para capitalização de palavras.

\usepackage{multicol}   % Permite incluir ambientes que exibem texto em múltiplas colunas.

\usepackage{xparse}     % Permite a definição de comandos com argumentos opcionais.

% -----

