% --- Português ---
\begin{resumoumacoluna}

    % Escreva aqui o resumo em português.
    \lipsum[1-3] % chktex-file 8

    \vspace{\onelineskip}
    \noindent
    \textbf{Palavras-chave}:
    % As palavras-chave em português devem ser definidas no arquivo de pretexto. Elas serão separadas pelo caractere ponto e vírgula.
    \ValorDasPalavrasChave{}.
\end{resumoumacoluna}

% --- Língua estrangeira ---
\renewcommand{\resumoname}{Abstract}
\begin{resumoumacoluna}
    \begin{otherlanguage*}{english}

        % Escreva aqui o resumo em língua estrangeira.
        \lipsum[4-6]

        \vspace{\onelineskip}
        \noindent
        \textbf{Keywords}:
        % Escreva aqui as palavras-chave em língua estrangeira.
        word one; word two; word three; word four; word five.
    \end{otherlanguage*}
\end{resumoumacoluna}

% \begin{resumoumacoluna}[Résumé]
%     \begin{otherlanguage*}{french}
%         Il s'agit d'un résumé en français.

%         \vspace{\onelineskip}
%         \noindent
%         \textbf{Mots-clés}:
%         latex; abntex; publication de textes.
%     \end{otherlanguage*}
% \end{resumoumacoluna}

% \begin{resumoumacoluna}[Resumen]
%     \begin{otherlanguage*}{spanish}
%         Este es el resumen en español.

%         \vspace{\onelineskip}
%         \noindent
%         \textbf{Palabras clave}: latex; abntex; publicación de textos.
%     \end{otherlanguage*}
% \end{resumoumacoluna}
