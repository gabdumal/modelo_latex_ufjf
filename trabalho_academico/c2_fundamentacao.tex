\chapter{Fundamentação Teórica}%
\label{cap:fundamentacao}

Apresentamos exemplos de uso de símbolos matemáticos e equações.

\section{Dados}

\subsection{Índices}

O problema permite delimitar os seguintes índices:

\begin{symbols}
    \item[\( \gls{indice:receitas} \)]
    \glsentrydesc{indice:receitas},
    tal que \(  \gls{indice:receitas} \in [1, N] \);

    \item[\( \gls{indice:pacotes} \)]
    \glsentrydesc{indice:pacotes},
    tal que \(  \gls{indice:pacotes} \in [1, N] \);

    \item[\( \gls{indice:pedidos} \)]
    \glsentrydesc{indice:pedidos},
    tal que \(  \gls{indice:pedidos} \in [1, N] \);
\end{symbols}

\subsection{Constantes}

\subsubsection{Fixas}

As seguinte constante é fixa e não pode ser alterada durante a execução do modelo.

\begin{symbols}
    \item[\( \gls{constante:fixa:tempo-turno} \) ]
    \glsentrydesc{constante:fixa:tempo-turno},
    tal que \(  \gls{constante:fixa:tempo-turno} \in \mathbb{N}^{+} \).
\end{symbols}

O valor da constante citada e sua unidade são exibidos no \autoref{qua:constantes-fixas}.

\begin{quadro}
    \caption{%
        \label{qua:constantes-fixas}%
        Constantes do problema.
    }

    \begin{tabular}{|c|c|c|}
        \hline
        Constante                              &
        Valor                                  &
        Unidade
        \\
        \hline
        \( \gls{constante:fixa:tempo-turno} \) &
        420                                    &
        minutos
        \\
        \hline
    \end{tabular}

    \ComponenteFontePropria{}
\end{quadro}

\subsubsection{Dados}

\begin{symbols}
    \item[\( \gls{constante:dado:receita:rendimento} \)]
    \glsentrydesc{constante:dado:receita:rendimento},
    tal que \( \glsentryname{constante:dado:receita:rendimento} \in \mathbb{R}^{+} \);

    \item[\( \gls{constante:dado:receita:rendimento-total} \)]
    \glsentrydesc{constante:dado:receita:rendimento-total},
    tal que \( \glsentryname{constante:dado:receita:rendimento-total} \in \mathbb{R}^{+} \);

    \item[\( \gls{constante:dado:receita:tempo-mistura} \)]
    \glsentrydesc{constante:dado:receita:tempo-mistura},
    tal que \( \glsentryname{constante:dado:receita:tempo-mistura} \in \mathbb{N}^{+} \);

    \item[\( \gls{constante:dado:pacote:gramatura} \)]
    \glsentrydesc{constante:dado:pacote:gramatura},
    tal que \( \glsentryname{constante:dado:pacote:gramatura} \in \mathbb{N}^{+} \);

    \item[\( \gls{constante:dado:pedido:peso} \)]
    \glsentrydesc{constante:dado:pedido:peso},
    tal que \( \glsentryname{constante:dado:pedido:peso} \in \mathbb{R}^{+} \).
\end{symbols}

\subsection{Variáveis}

\subsubsection{Auxiliares}

Todas as variáveis auxiliares acrescentam uma coluna à matriz de dados, sendo calculadas a partir de outras variáveis.

As variáveis auxiliares calculadas a partir do problema são:

\begin{symbols}
    \item[\( \gls{variavel:auxiliar:peso-total-pedidos-atendidos} \)]
    \glsentrydesc{variavel:auxiliar:peso-total-pedidos-atendidos},
    tal que \( \glsentryname{variavel:auxiliar:peso-total-pedidos-atendidos} \in \mathbb{R} \);

    \item[\( \gls{variavel:auxiliar:peso-total-produzido} \)]
    \glsentrydesc{variavel:auxiliar:peso-total-produzido},
    tal que \( \glsentryname{variavel:auxiliar:peso-total-produzido} \in \mathbb{R} \);

    \item[\( \gls{variavel:auxiliar:peso-total-em-sobra} \)]
    \glsentrydesc{variavel:auxiliar:peso-total-em-sobra},
    tal que \( \glsentryname{variavel:auxiliar:peso-total-em-sobra} \in \mathbb{R} \);
\end{symbols}

\subsubsection{Principais}

\begin{symbols}
    \item[\( \gls{variavel:principal:quantidade-porcoes-receita-a-produzir} \)]
    \glsentrydesc{variavel:principal:quantidade-porcoes-receita-a-produzir},
    tal que \(  \glsentryname{variavel:principal:quantidade-porcoes-receita-a-produzir} \in \mathbb{N} \).
\end{symbols}

\section{Restrições}

\subsection{Auxiliares}

Todas as variáveis auxiliares incorrem em restrições de igualdade, uma vez que são calculadas segundo colunas da matriz de dados.

As restrições de igualdade que definem as variáveis auxiliares são:

\begin{align}
    \gls{variavel:auxiliar:peso-total-produzido} &
    = \gls{variavel:principal:quantidade-porcoes-receita-a-produzir} \times \gls{constante:dado:receita:rendimento}.
    \\
    \gls{variavel:auxiliar:peso-total-em-sobra}  &
    = \gls{variavel:auxiliar:peso-total-produzido} - \gls{variavel:auxiliar:peso-total-pedidos-atendidos}.
    \\
    \sum \gls{constante:dado:rendimento-total}   &
    \geq
    \sum \gls{constante:dado:pedido:peso}
    \\
    \gls{constante:dado:receita:tempo-mistura}   &
    \leq 420
\end{align}

Tempo máximo de mistura por expediente:

\begin{align}
    \gls{variavel:principal:quantidade-porcoes-receita-a-produzir}
    \times \gls{constante:dado:receita:tempo-mistura}
    \leq \gls{constante:fixa:tempo-turno}.
\end{align}

\section{Função-Objetivo}

A função objetivo deste problema é minimizar o peso total em sobra das receitas, conforme mostrado abaixo:

\begin{align}
    \min \sum \gls{variavel:auxiliar:peso-total-em-sobra}_{\gls{indice:pedidos}}
\end{align}
