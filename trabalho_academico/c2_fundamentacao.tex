\chapter{Fundamentação Teórica}%
\label{cap:fundamentacao}

Apresentamos exemplos de uso de símbolos matemáticos e equações.

\section{Dados}

\subsection{Índices}

O problema permite delimitar os seguintes índices:

\begin{symbols}
    \item[\( \gls{indice:receitas} \)]
    \glsentrydesc{indice:receitas},
    tal que \(  \gls{indice:receitas} \in [1, \gls{constante:pre-processada:quantidade-receitas}] \);

    \item[\( \gls{indice:pacotes} \)]
    \glsentrydesc{indice:pacotes},
    tal que \(  \gls{indice:pacotes} \in [1, \gls{constante:pre-processada:quantidade-pacotes}] \);

    \item[\( \gls{indice:pedidos} \)]
    \glsentrydesc{indice:pedidos},
    tal que \(  \gls{indice:pedidos} \in [1, \gls{constante:pre-processada:quantidade-pedidos}] \);
\end{symbols}

\subsection{Constantes}

\subsubsection{Fixas}

As seguinte constante é fixa e não pode ser alterada durante a execução do modelo.

\begin{symbols}
    \item[\( \gls{constante:fixa:tempo-turno} \) ]
    \glsentrydesc{constante:fixa:tempo-turno},
    tal que \(  \gls{constante:fixa:tempo-turno} \in \mathbb{N}^{+} \).
\end{symbols}

O valor da constante citada e sua unidade são exibidos no \autoref{qua:constantes-fixas}.

\begin{quadro}
    \caption{%
        \label{qua:constantes-fixas}%
        Constantes do problema.
    }

    \begin{tabular}{|c|c|c|}
        \hline
        Constante                              &
        Valor                                  &
        Unidade
        \\
        \hline
        \( \gls{constante:fixa:tempo-turno} \) &
        420                                    &
        minutos
        \\
        \hline
    \end{tabular}

    \ComponenteFontePropria{}
\end{quadro}

\subsubsection{Pré-processadas}

Alguns dados preenchidos na aplicação web, apesar de serem dinâmicos, podem ser calculados antes da execução do modelo, permanecendo inalterados durante a execução do algoritmo de otimização.

Na matriz de dados, essas constantes são representados na forma de restrições de igualdade calculadas a partir de outras variáveis.
Do lado esquerdo da equação, tem-se o número constante pré-calculado e, do lado direito, a soma de variáveis multiplicadas por coeficientes.

Segue-se a lista de constantes pré-processadas:

\begin{symbols}
    \item[\( \gls{constante:pre-processada:quantidade-receitas} \)]
    \glsentrydesc{constante:pre-processada:quantidade-receitas},
    tal que \( \glsentryname{constante:pre-processada:quantidade-receitas} \in \mathbb{N}^{+} \);

    \item[\( \gls{constante:pre-processada:quantidade-pacotes} \)]
    \glsentrydesc{constante:pre-processada:quantidade-pacotes},
    tal que \( \glsentryname{constante:pre-processada:quantidade-pacotes} \in \mathbb{N}^{+} \);

    \item[\( \gls{constante:pre-processada:quantidade-pedidos} \)]
    \glsentrydesc{constante:pre-processada:quantidade-pedidos},
    tal que \( \glsentryname{constante:pre-processada:quantidade-pedidos} \in \mathbb{N}^{+} \).
\end{symbols}

\subsubsection{Dados}

\begin{symbols}
    \item[\( \gls{constante:dado:receita:rendimento} \)]
    \glsentrydesc{constante:dado:receita:rendimento},
    tal que \( \glsentryname{constante:dado:receita:rendimento} \in \mathbb{R}^{+} \);

    \item[\( \gls{constante:dado:receita:rendimento-total} \)]
    \glsentrydesc{constante:dado:receita:rendimento-total},
    tal que \( \glsentryname{constante:dado:receita:rendimento-total} \in \mathbb{R}^{+} \);

    \item[\( \gls{constante:dado:receita:tempo-mistura} \)]
    \glsentrydesc{constante:dado:receita:tempo-mistura},
    tal que \( \glsentryname{constante:dado:receita:tempo-mistura} \in \mathbb{N}^{+} \);

    \item[\( \gls{constante:dado:receita:tempo-modelagem} \)]
    \glsentrydesc{constante:dado:receita:tempo-modelagem},
    tal que \( \glsentryname{constante:dado:receita:tempo-modelagem} \in \mathbb{N}^{+} \);

    \item[\( \gls{constante:dado:receita:tempo-assamento} \)]
    \glsentrydesc{constante:dado:receita:tempo-assamento},
    tal que \( \glsentryname{constante:dado:receita:tempo-assamento} \in \mathbb{N}^{+} \);

    \item[\( \gls{constante:dado:pacote:gramatura} \)]
    \glsentrydesc{constante:dado:pacote:gramatura},
    tal que \( \glsentryname{constante:dado:pacote:gramatura} \in \mathbb{N}^{+} \);

    \item[\( \gls{constante:dado:pedido:quantidade-pacotes} \)]
    \glsentrydesc{constante:dado:pedido:quantidade-pacotes},
    tal que \( \glsentryname{constante:dado:pedido:quantidade-pacotes} \in \mathbb{N}^{+} \).

    \item[\( \gls{constante:dado:pedido:peso} \)]
    \glsentrydesc{constante:dado:pedido:peso},
    tal que \( \glsentryname{constante:dado:pedido:peso} \in \mathbb{R}^{+} \).
\end{symbols}

\subsection{Variáveis}

\subsubsection{Auxiliares}

Todas as variáveis auxiliares acrescentam uma coluna à matriz de dados, sendo calculadas a partir de outras variáveis.

Isso ocorre pois, na forma padrão de um problema de \gls{pl}, toda restrição tem, no membro esquerdo, uma soma de variáveis multiplicadas por coeficientes e, no membro direito, um valor constante.

Assim, na restrição de igualdade que define uma variável auxiliar, se move a variável para o membro direito da equação e inverte o sinal do coeficiente.

As variáveis auxiliares calculadas a partir do problema são:

\begin{symbols}
    \item[\( \gls{variavel:auxiliar:peso-total-pedidos-atendidos} \)]
    \glsentrydesc{variavel:auxiliar:peso-total-pedidos-atendidos},
    tal que \( \glsentryname{variavel:auxiliar:peso-total-pedidos-atendidos} \in \mathbb{R} \);

    \item[\( \gls{variavel:auxiliar:peso-total-produzido} \)]
    \glsentrydesc{variavel:auxiliar:peso-total-produzido},
    tal que \( \glsentryname{variavel:auxiliar:peso-total-produzido} \in \mathbb{R} \);

    \item[\( \gls{variavel:auxiliar:peso-total-em-sobra} \)]
    \glsentrydesc{variavel:auxiliar:peso-total-em-sobra},
    tal que \( \glsentryname{variavel:auxiliar:peso-total-em-sobra} \in \mathbb{R} \);
\end{symbols}

\subsubsection{Principais}

\begin{symbols}
    \item[\( \gls{variavel:principal:deve-atender-pedido} \)]
    \glsentrydesc{variavel:principal:deve-atender-pedido},
    tal que \(  \glsentryname{variavel:principal:deve-atender-pedido} \in \{0, 1\} \).

    \item[\( \gls{variavel:principal:quantidade-porcoes-receita-a-produzir} \)]
    \glsentrydesc{variavel:principal:quantidade-porcoes-receita-a-produzir},
    tal que \(  \glsentryname{variavel:principal:quantidade-porcoes-receita-a-produzir} \in \mathbb{N} \).
\end{symbols}

\section{Restrições}

\subsection{Constantes pré-processadas}

As constantes pré-processadas, que se definem intrinsecamente por sua relação com variáveis, são:

\begin{align}
    \gls{constante:pre-processada:quantidade-receitas} &
    = \sum_{\gls{indice:receitas} = 1}^{\gls{constante:pre-processada:quantidade-receitas}} 1,
    \\
    \gls{constante:pre-processada:quantidade-pacotes}  &
    = \sum_{\gls{indice:pacotes} = 1}^{\gls{constante:pre-processada:quantidade-pacotes}} 1,
    \\
    \gls{constante:pre-processada:quantidade-pedidos}  &
    = \sum_{\gls{indice:pedidos} = 1}^{\gls{constante:pre-processada:quantidade-pedidos}} 1.
\end{align}

\subsection{Auxiliares}

Todas as variáveis auxiliares incorrem em restrições de igualdade, uma vez que são calculadas segundo colunas da matriz de dados.

As restrições de igualdade que definem as variáveis auxiliares são:

\begin{align}
    \gls{variavel:auxiliar:peso-total-pedidos-atendidos} &
    = \frac{\sum_{\gls{indice:pedidos} = 1}^{\gls{constante:pre-processada:quantidade-pedidos}} (\gls{constante:dado:pedido:quantidade-pacotes} \times \gls{constante:dado:pacote:gramatura})}
    {1000}.
    \\
    \gls{variavel:auxiliar:peso-total-produzido}         &
    = \gls{variavel:principal:quantidade-porcoes-receita-a-produzir} \times \gls{constante:dado:receita:rendimento}.
    \\
    \gls{variavel:auxiliar:peso-total-em-sobra}          &
    = \gls{variavel:auxiliar:peso-total-produzido} - \gls{variavel:auxiliar:peso-total-pedidos-atendidos}.
    \\
    \sum \gls{constante:dado:rendimento-total}           &
    \geq
    \sum \gls{constante:dado:pedido:peso}
    \\
    \gls{constante:dado:receita:tempo-mistura}           &
    \leq 420
    \\
    \gls{constante:dado:receita:tempo-modelagem}         &
    \leq 420
    \\
    \gls{constante:dado:receita:tempo-assamento}         &
    \leq 405
\end{align}

\subsection{Demais restrições}

Toda a demanda deve ser atendida:

\begin{align}
    \gls{variavel:auxiliar:peso-total-produzido} &
    \geq \sum_{\gls{indice:pedidos} = 1}
    ^{\gls{constante:pre-processada:quantidade-pedidos}} \gls
    {constante:dado:pedido:quantidade-pacotes}.
\end{align}

Tempo máximo de mistura por expediente:

\begin{align}
    \gls{variavel:principal:quantidade-porcoes-receita-a-produzir}
    \times \gls{constante:dado:receita:tempo-mistura}
    \leq \gls{constante:fixa:tempo-turno}.
\end{align}

\section{Função-Objetivo}

A função objetivo deste problema é minimizar o peso total em sobra das receitas, conforme mostrado abaixo:

\begin{align}
    \min \sum \gls{variavel:auxiliar:peso-total-em-sobra}_{\gls{indice:pedidos}}
\end{align}
