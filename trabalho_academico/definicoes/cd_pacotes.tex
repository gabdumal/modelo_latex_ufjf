% -----
% Pacotes fundamentais 
% -----

\usepackage{lmodern}        % Usa a fonte Latin Modern.
\usepackage[T1]{fontenc}    % Seleção de códigos de fonte.
\usepackage[utf8]{inputenc} % Codificação do documento (conversão automática dos acentos).
\usepackage{indentfirst}    % Identa o primeiro parágrafo de cada seção.
\usepackage{microtype}      % Para melhorias de justificação.

\usepackage{color}      % Controle das cores.
\usepackage{graphicx}   % Inclusão de gráficos.

\usepackage[brazilian,hyperpageref]{backref}    % Páginas com as citações na bibliografia.

\usepackage[
    abbreviations,
    symbols,
    nohypertypes={index},
    % nonumberlist=true,
    seeautonumberlist,
    subentrycounter,
    toc=false
]{glossaries-extra}   % Glossário. Permite definir termos, siglas e abreviações. Deve ser carregado depois de backref.

\usepackage[alf]{abntex2cite}       % Citações padrão ABNT. Deve ser carregado depois de glossaries.

\usepackage[portuguese]{todonotes}  % Adiciona notas e afazeres no documento.

\usepackage{multirow}   % Permite mesclar células em tabelas.
\usepackage{longtable}  % Permite criar tabelas com mais de uma página.

\usepackage{lipsum}     % Gera texto de preenchimento.

\usepackage{titlecaps}  % Fornece ferramentas para capitalização de palavras.

\usepackage{multicol}   % Permite incluir ambientes que exibem texto em múltiplas colunas.

% -----

% -----
% Configurações dos pacotes
% -----

% --- Glossaries-extra ---
% O pacote "glossaries-extra" provê a opção de criar um glossário chamado "index" nas opções do pacote. Entretanto, isso cria um comando chamado "\printindex", que conflita com o homônimo do pacote "memoir", utilizado para imprimir o índice remissivo. Para resolver, não habilitamos a opção "index" e criamos um novo glossário chamado "index" manualmente.
\newglossary[ilg]{index}{ind}{idx}{\indexname}
% Define o comando "\newterm" para facilitar a criação de novos termos no glossário "index".
\DeclareDocumentCommand\newterm{m m g}{%
    \IfValueTF{#3}{%
        \newglossaryentry{#1}{%
            type=index,
            name={#2},
            description={\nopostdesc},
            plural={#3}%
        }
    }{%
        \newglossaryentry{#1}{%
            type=index,
            name={#2},
            description={\nopostdesc}%
        }
    }
}
% Permite a impressão dos glossários.
\makeglossaries{}
\loadglsentries{c0_glossario.tex}
\renewcommand*{\glsclearpage}{} % Evita quebra de página entre os glossários.

% --- Backref ---
% Usado sem a opção hyperpageref de backref.
\renewcommand{\backrefpagesname}{Citado na(s) página(s):~} % chktex-file 36
% Texto padrão antes do número das páginas.
\renewcommand{\backref}{}
% Define os textos da citação.
\renewcommand*{\backrefalt}[4]{%
    \ifcase #1% chktex-file 1
        Nenhuma citação no texto.%
    \or%
        Citado na página #2.%
    \else%
        Citado #1 vezes nas páginas #2.%
    \fi}%

% --- Todonotes ---
% Define a largura da caixa de notas.
\setlength{\marginparwidth}{2cm}
% Define a cor e o estilo da caixa de notas.
\presetkeys{todonotes}{inline,backgroundcolor=yellow}{}
% Desabilita as notas.
% \presetkeys{todonotes}{disable}{}

% --- Titlecaps ---
% Define as palavras que não devem ser capitalizadas. Elas ainda serão capitalizadas se estiverem no início de uma frase.
\Addlcwords{a o e um uma uns umas em de da do das dos no na nos nas num numa nuns numas por para com pelo pela pelos pelas}

% -----
