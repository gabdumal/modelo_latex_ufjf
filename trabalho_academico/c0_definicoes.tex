% =====
% Definições para gerar os artefatos
% =====

% -----
% Título 
% -----

% Título principal do trabalho. Parâmetros: título.
\titulo{Título do meu trabalho}

% Subtítulo do trabalho. Parâmetros: subtítulo (opcional).
\subtitulo{O subtítulo do meu trabalho}

% -----

% -----
% Trabalho
% -----

% Tipo de trabalho. Parâmetros: classificação (doutorado, mestrado, especializacao, licenciatura, bacharelado).
\tipo{bacharelado}

% Curso. Parâmetros: nome do curso.
\curso{Sistemas de Informação}

% Data de aprovação. Parâmetros: dia; mês por extenso, em letras minúsculas; ano no formato 1000.
\dataDeAprovacao{01}{janeiro}{2025}

% --- Local de publicação ---

% Cidade. Parâmetros: nome da cidade.
\cidade{Juiz de Fora}

% Estado (Opcional). Parâmetros: nome do estado; sigla do estado.
% \estado{Minas Gerais}{MG}

% País. Parâmetros: nome do país.
\pais{Brasil}

% -----

% -----
% Instituição 
% -----

% Instituição. Parâmetros: artigo gramatical do gênero (a,o); nome; sigla; cidade.
\instituicao{a}{Universidade Federal de Juiz de Fora}{UFJF}{Juiz de Fora}

% Unidade acadêmica. Parâmetros: artigo gramatical do gênero (a,o); nome; sigla.
\unidadeAcademica{o}{Instituto de Ciências Exatas}{ICE}

% Programa. Apenas necessário preencher para trabalhos de mestrado e doutorado. Parâmetros: artigo gramatical do gênero (a,o); nome; sigla.
\programa{o}{Programa de Pós-Graduação em Ciência da Computação}{PPGCC}

% -----

% -----
% Pessoas 
% -----

% Autor. Parâmetros: Último sobrenome; restante do nome.
\autor{Alves}{Alice Carvalho de}

% Orientador. Parâmetros: gênero (feminino,masculino); título; último sobrenome; restante do nome; instituição de afiliação.
\orientador%
{feminino}
{Profa.\ Dra.}
{Batista}
{Beatriz Alves}
{Universidade Federal de Juiz de Fora}

% Coorientador. Parâmetros: gênero (feminino,masculino); título; último sobrenome; restante do nome; instituição de afiliação.
\coorientador%
{masculino}
{Prof.\ Dr.}
{Carvalho}
{Carlos Ribeiro de}
{Universidade Federal de Juiz de Fora}

% Examinador um. Parâmetros: título; último sobrenome; restante do nome; instituição de afiliação.
\examinadorUm%
{Prof.\ Dr.}
{Dias}
{Daniel Pereira}
{Universidade Federal de Juiz de Fora}

% Examinador dois. Parâmetros: título; último sobrenome; restante do nome; instituição de afiliação.
\examinadorDois%
{Profa.\ Dra.}
{Espinoza}
{Eva Maria}
{Universidade Federal de Juiz de Fora}

% % Examinador três. Parâmetros: título; último sobrenome; restante do nome; instituição de afiliação.
% \examinadorTres%
% {Prof.\ Dr.}
% {Ferreira}
% {Fernando dos Santos}
% {Universidade Federal de Juiz de Fora}

% % Examinador quatro. Parâmetros: título; último sobrenome; restante do nome; instituição de afiliação.
% \examinadorQuatro%
% {Profa.\ Dra.}
% {Gonçalves}
% {Gabriela Almeida}
% {Universidade Federal de Juiz de Fora}

% -----

% =====
