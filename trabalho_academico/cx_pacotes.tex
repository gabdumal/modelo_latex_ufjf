% -----
% Pacotes fundamentais 
% -----

\usepackage{lmodern}        % Usa a fonte Latin Modern.
\usepackage[T1]{fontenc}    % Seleção de códigos de fonte.
\usepackage[utf8]{inputenc} % Codificação do documento (conversão automática dos acentos).
\usepackage{indentfirst}    % Identa o primeiro parágrafo de cada seção.
\usepackage{microtype}      % Para melhorias de justificação.

\usepackage{color}      % Controle das cores.
\usepackage{graphicx}   % Inclusão de gráficos.

\usepackage[brazilian,hyperpageref]{backref}    % Páginas com as citações na bibliografia.

\usepackage[subentrycounter,seeautonumberlist,nonumberlist=true,acronym,nohypertypes={index}]{glossaries}   % Glossário. Permite definir termos, siglas e abreviações. Deve ser carregado depois de backref.

\usepackage[alf]{abntex2cite}       % Citações padrão ABNT. Deve ser carregado depois de glossaries.

\usepackage[portuguese]{todonotes}  % Adiciona notas e afazeres no documento.

\usepackage{multirow}   % Permite mesclar células em tabelas.
\usepackage{longtable}  % Permite criar tabelas com mais de uma página.

\usepackage{lipsum}     % Gera texto de preenchimento.

% -----

% -----
% Configurações dos pacotes
% -----

% --- Glossaries ---
\newglossary[ilg]{index}{ind}{idx}{\indexname}
\newcommand*{\newterm}[2]{
    \newglossaryentry{#1}
    {type=index,name={#2},description={\nopostdesc}}
}
\makeglossaries{} % Habilite este comando para permitir a impressão dos glossários.
\loadglsentries{c0_glossario.tex}
\renewcommand*{\glsclearpage}{} % Evita quebra de página entre os glossários.

% --- Backref ---
% Usado sem a opção hyperpageref de backref.
\renewcommand{\backrefpagesname}{Citado na(s) página(s):~}
% Texto padrão antes do número das páginas.
\renewcommand{\backref}{}
% Define os textos da citação.
\renewcommand*{\backrefalt}[4]{%
    \ifcase #1%
        Nenhuma citação no texto.%
    \or%
        Citado na página #2.%
    \else%
        Citado #1 vezes nas páginas #2.%
    \fi}%

% --- Todonotes ---
% Define a largura da caixa de notas.
\setlength{\marginparwidth}{2cm}
% Define a cor e o estilo da caixa de notas.
\presetkeys{todonotes}{inline,backgroundcolor=yellow}{}
% Desabilita as notas.
% \presetkeys{todonotes}{disable}{}

% -----