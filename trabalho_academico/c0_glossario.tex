% -----
% Termos simples
% -----

\newterm{disciplina}{DCC067 Computação Evolucionista}
\newterm{pl}{Programação Linear}
\newterm{fn}{função}{funções}
% \newglossaryentry{fn}{
%     name={função},
%     plural={funções},
%     % Essa opção define que a entrada não apresenta descrição.
%     description={\nopostdesc},
%     % Essa opção categoriza a entrada no glossário de termos simples.
%     type=index
% }

% -----

% -----
% Acrônimos, siglas e abreviações
% -----

\newabbreviation{dcc}{DCC}{Departamento de Ciência da Computação}

\newabbreviation{ice}{ICE}{Instituto de Ciências Exatas}

\newabbreviation{ufjf}{UFJF}{Universidade Federal de Juiz de Fora}

\newabbreviation[
    description={Algoritmo Genético, do inglês \textit{Genetic Algorithm}, é uma técnica de otimização baseada em evolução biológica},
    plural={GAs},
    longplural={Algoritmos Genéticos}
]{ga}{GA}{Algoritmo Genético}

% -----

% -----
% Definições
% -----

\newglossaryentry{crossover}{
    name={Crossover},
    text={\textit{crossover}},
    description={Operador genético que combina dois indivíduos para gerar um ou mais descendentes}
}

\newglossaryentry{fitness}{
    name={Fitness},
    text={\textit{fitness}},
    description={Medida de qualidade de um indivíduo em um algoritmo evolutivo}
}

% -----

% -----
% Símbolos
% -----

\glsxtrnewsymbol[
    description={Conjunto vazio, que não contém elementos}
]
{conjunto_vazio}{\ensuremath{
        \emptyset
    }}

% --- Índices ---

\glsxtrnewsymbol[
    description={Índice das receitas disponíveis no catálogo}
]
{indice:receitas}{\ensuremath{{
                \textcolor{teal}{ \boldsymbol{r}}
            }}}

\glsxtrnewsymbol[
    description={Índice dos pacotes disponíveis para acondicionamento de produtos, categorizados pela sua gramatura}
]
{indice:pacotes}{\ensuremath{{
                \textcolor{teal}{ \boldsymbol{e} }
            }}}

\glsxtrnewsymbol[
    description={Índice dos pedidos realizados por clientes no período analisado}
]
{indice:pedidos}{\ensuremath{{
                \textcolor{teal}{ \boldsymbol{p} }
            }}}

% --- Constantes ---

% -- Fixas --

\glsxtrnewsymbol[
    description={Tempo total de disponibilidade do maquinário em expediente}
]
{constante:fixa:tempo-turno}{\ensuremath{{
                \textcolor{brown}{ \boldsymbol{T_{T}} }
            }}}

% -- Dados --

% - Básicos -

\glsxtrnewsymbol[
    description={Rendimento (em quilogramas)}
]
{constante:dado:rendimento}{\ensuremath{{
                \textcolor{brown}{ \boldsymbol{R} }
            }}}

\glsxtrnewsymbol[
    description={Tempo (em minutos)}
]
{constante:dado:tempo}{\ensuremath{{
                \textcolor{brown}{ \boldsymbol{T} }
            }}}

\glsxtrnewsymbol[
    description={Peso (em quilogramas)}
]
{constante:dado:peso}{\ensuremath{{
                \textcolor{brown}{ \boldsymbol{M} }
            }}}

\glsxtrnewsymbol[
    description={Categorização de pacotes por gramatura (em gramas)}
]
{constante:dado:gramatura}{\ensuremath{{
                \textcolor{brown}{ \boldsymbol{E} }
            }}}

% - Derivados -

\glsxtrnewsymbol[
    description={Rendimento total (em quilogramas)}
]
{constante:dado:rendimento-total}{\ensuremath{{
                \textcolor{brown}{ \boldsymbol{R_{T}} }
            }}}

\glsxtrnewsymbol[
description={Tempo de mistura (batedeira) (em minutos)}
]
{constante:dado:tempo-mistura}{\ensuremath{{
\glsentryname{constante:dado:tempo}_{\textcolor{brown}{ \boldsymbol{B} }}
}}}

% - Indexados -

\glsxtrnewsymbol[
description={Rendimento de uma porção fabricada da receita \( \glsentryname{indice:receitas} \) (em quilogramas)}
]
{constante:dado:receita:rendimento}{\ensuremath{{
\glsentryname{constante:dado:rendimento}_{\glsentryname{indice:receitas}}
}}}

\glsxtrnewsymbol[
description={Rendimento total das porções fabricadas da receita \( \glsentryname{indice:receitas} \) (em quilogramas)}
]
{constante:dado:receita:rendimento-total}{\ensuremath{{
\glsentryname{constante:dado:rendimento-total}_{\glsentryname{indice:receitas}}
}}}

\glsxtrnewsymbol[
description={Tempo de mistura (batedeira) da receita \( \glsentryname{indice:receitas} \) (em minutos)}
]
{constante:dado:receita:tempo-mistura}{\ensuremath{{
\glsentryname{constante:dado:tempo-mistura}_{\glsentryname{indice:receitas}}
}}}

\glsxtrnewsymbol[
description={Quantidade (gramatura) necessária de uma receita para preencher um pacote \( \glsentryname{indice:pacotes} \) (em gramas)}
]
{constante:dado:pacote:gramatura}{\ensuremath{{
\glsentryname{constante:dado:gramatura}_{\glsentryname{indice:pacotes}}
}}}

\glsxtrnewsymbol[
description={Peso total de um pedido \( \glsentryname{indice:pedidos} \) (em quilogramas)}
]
{constante:dado:pedido:peso}{\ensuremath{{
\glsentryname{constante:dado:peso}_{\glsentryname{indice:pedidos}}
}}}

% --- Variáveis ---

% -- Auxiliares --

\glsxtrnewsymbol[
    description={Peso somado (demanda) dos pedidos de uma receita \( \glsentryname{indice:receitas} \) selecionados
            para serem atendidos (em quilogramas)}
]
{variavel:auxiliar:peso-total-pedidos-atendidos}{\ensuremath{{
                {\textcolor{purple}{ \boldsymbol{M_{A}} }}_{\glsentryname{indice:receitas}}
            }}}

\glsxtrnewsymbol[
    description={Peso total produzido de uma receita \( \glsentryname{indice:receitas} \) (em quilogramas)}
]
{variavel:auxiliar:peso-total-produzido}{\ensuremath{{
                {\textcolor{purple}{ \boldsymbol{M_{P}} }}_{\glsentryname{indice:receitas}}
            }}}

\glsxtrnewsymbol[
    description={Peso total de uma receita \( \glsentryname{indice:receitas} \) que sobrou para o estoque (em quilogramas)}
]
{variavel:auxiliar:peso-total-em-sobra}{\ensuremath{{
                {\textcolor{purple}{ \boldsymbol{M_{E}} }}_{\glsentryname{indice:receitas}}
            }}}

% -- Principais --

\glsxtrnewsymbol[
    description={Quantidade de porções (fornadas) de uma receita \( \glsentryname{indice:receitas} \) a serem produzidas}
]
{variavel:principal:quantidade-porcoes-receita-a-produzir}{\ensuremath{{
                \textcolor{blue}{ \boldsymbol{F} }_{\glsentryname{indice:receitas}}
            }}}

% -----
